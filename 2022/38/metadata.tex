% DO NOT EDIT - automatically generated from metadata.yaml

\def \codeURL{https://github.com/Bentgm17/Fact2022_27.git}
\def \codeDOI{}
\def \codeSWH{swh:1:dir:1cb55b0f026b758df313fea777575fcab71222f5}
\def \dataURL{}
\def \dataDOI{}
\def \editorNAME{Koustuv Sinha,\\ Maurits Bleeker,\\ Samarth Bhargav}
\def \editorORCID{}
\def \reviewerINAME{}
\def \reviewerIORCID{}
\def \reviewerIINAME{}
\def \reviewerIIORCID{}
\def \dateRECEIVED{04 February 2023}
\def \dateACCEPTED{19 April 2023}
\def \datePUBLISHED{20 July 2023}
\def \articleTITLE{[Re] CrossWalk Fairness-enhanced Node Representation Learning}
\def \articleTYPE{Replication}
\def \articleDOMAIN{ML Reproducibility Challenge 2022}
\def \articleBIBLIOGRAPHY{bibliography.bib}
\def \articleYEAR{2023}
\def \reviewURL{https://openreview.net/forum?id=KNp7Zq3KkT0&noteId=xqiAvezWIN}
\def \articleABSTRACT{'CrossWalk Fairness-Enhanced Node Representation Learning' is set to be reproduced and reviewed. It presents an extension to existing graph algorithms that incorporate the idea of biased random walks for obtaining node embeddings. CrossWalk incorporates fairness by up-weighting edges of nodes located near group boundaries. The authors claim that their approach outperforms baseline algorithms, such as DeepWalk and FairWalk, in terms of reducing the disparity between different classes within a graph network. The authors accompanied their paper with the publication of an open GitHub page, which includes the source code and relevant data sets. The limited size of the data sets in combination with the efficient algorithms enables the experiments to be conducted without significant difficulties and is computable on standard CPUs without the need for additional resources. In this reproducibility report, the outcomes of the experiments are in agreement with the results presented in the original paper. However, the inherent randomness of the random walks makes it difficult to quantify the extent of similarity between the reproduced results and the results as stated in the original paper. However, it can be concluded that CrossWalk results in a decreased disparity between groups in graph networks. The authors effectively conveyed the underlying concept of their proposed method, rendering it both intriguing and straightforward to comprehend the key ideas. Furthermore, the authors successfully incorporated a range of methods and baseline algorithms into the paper. In contrast, the source code may not have been optimally constructed with reproducibility in mind. Certain sections of the code appear to be unfinished or inadequately executed. Additionally, the authors neglected to specify key hyperparameters, resulting in the unidentifiability of certain results. This presents challenges in drawing conclusions based on the available sources. The authors were unable to respond in time for elaborating on certain implementation details. However, we did receive additional data which was crucial to obtaining certain results.}
\def \replicationCITE{A. Khajehnejad, M. Khajehnejad, M. Babaei, K. P. Gummadi, A. Weller, and B. Mirzasoleiman. CrossWalk Fairness-enhanced node representation learning. Proceedings of the AAAI Conference on Artificial Intelligence. Vol. 36. 11. 2022,pp. 11963-11970.}
\def \replicationBIB{khajehnejad2022crosswalk}
\def \replicationURL{https://arxiv.org/pdf/2105.02725.pdf}
\def \replicationDOI{}
\def \contactNAME{Meggie van den Oever}
\def \contactEMAIL{megv.d.oever@gmail.com}
\def \articleKEYWORDS{rescience c, rescience x, machine learning, deep learning, python, fairness, graphs, node representation, crosswalk, deepwalk, moenswalk, graph networks}
\def \journalNAME{ReScience C}
\def \journalVOLUME{9}
\def \journalISSUE{2}
\def \articleNUMBER{38}
\def \articleDOI{}
\def \authorsFULL{Gijs Joppe Moens et al.}
\def \authorsABBRV{G.J. Moens et al.}
\def \authorsSHORT{Moens et al.}
\title{\articleTITLE}
\date{}
\author[1,\orcid{0009-0008-8767-8629}]{Gijs Joppe Moens}
\author[1,\orcid{0000-0002-5028-5943}]{Job de Witte}
\author[1,\orcid{0009-0004-7022-0719}]{Tobias Pieter Göbel}
\author[1,\orcid{0009-0004-3243-2462}]{Meggie van den Oever}
\affil[1]{University of Amsterdam, Amsterdam, The Netherlands}
