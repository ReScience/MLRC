% DO NOT EDIT - automatically generated from metadata.yaml

\def \codeURL{https://github.com/mcleish7/MLRC-deep-thinking}
\def \codeDOI{}
\def \codeSWH{swh:1:dir:be1a763d08c1af34a2721fad6b9bfe8c99527220}
\def \dataURL{}
\def \dataDOI{}
\def \editorNAME{Koustuv Sinha,\\ Maurits Bleeker,\\ Samarth Bhargav}
\def \editorORCID{}
\def \reviewerINAME{}
\def \reviewerIORCID{}
\def \reviewerIINAME{}
\def \reviewerIIORCID{}
\def \dateRECEIVED{04 February 2023}
\def \dateACCEPTED{19 April 2023}
\def \datePUBLISHED{20 July 2023}
\def \articleTITLE{[Re] End-to-end Algorithm Synthesis with Recurrent Networks: Logical Extrapolation Without Overthinking}
\def \articleTYPE{Replication}
\def \articleDOMAIN{ML Reproducibility Challenge 2022}
\def \articleBIBLIOGRAPHY{bibliography.bib}
\def \articleYEAR{2023}
\def \reviewURL{https://openreview.net/forum?id=WaZB4pUVTi}
\def \articleABSTRACT{Scope of Reproducibility: In this report, we aim to validate the claims of Bansal et al. These are that the recurrent architecture presented, with skip connections and a progressive loss function, prevent the original problem being forgotten or corrupted during processing allowing for the recurrent module to be applied indefinitely and that this architecture avoids the overthinking trap. We use both code released by the authors and newly developed to recreate many results presented in the paper. Additionally, we present analysis of the newly introduced alpha hyperparameter and investigate interesting perturbation behaviour of prefix sums models. Further, we conduct a hyperparameter search and provide an analysis of the Asymptotic Alignment scores of the models presented.
Methodology: We use the PyTorch code released by the authors to replicate accuracy experiments. We then, independently, develop our own PyTorchFI to replicate perturbation experiments presented by Bansal et al. Overall, providing a replication of all results shown in the main body of the paper. We then extend these results, providing an analysis of the alpha hyperparameter, analysis of perturbation recovery, Asymptotic Alignment scores and a hyperparameter search. We used both a Nvidia RTX 2080Ti GPU and sets of three NVIDIA Quadro RTX6000 GPUs, taking a total of 982.2 GPU hours to create all results presented in the main body of this report.
Results: We verify the authors' claims by replicating the experiments presented by Bansal et al. All of our experiments show identical results to the ones presented in the paper, apart from perturbation testing for which we provide an additional in depth analysis. We also provide an analysis of the new alpha hyperparameter and a hyperparameter search.
What was easy: The code provided by Bansal et al gave clear instructions on how to use it, along with pretrained models being available for all problems.
What was difficult: Chess models required a considerable amount of time to train, putting a drain on resources. Also, code for reproducing perturbation results was not available so this had to developed from scratch.
Communication with original authors: We had good communication with the original authors, both emailing and meeting online.}
\def \replicationCITE{A. Bansal, A. Schwarzschild, E. Borgnia, Z. Emam, F. Huang, M. Goldblum, and T. Goldstein. “End-to-end Algorithm Synthesis with Recurrent Networks: Logical Extrapolation Without Overthinking.” In: Advances in Neural Information Processing Systems 35 (2022)}
\def \replicationBIB{bansal2022endtoend}
\def \replicationURL{https://arxiv.org/pdf/2202.05826.pdf}
\def \replicationDOI{https://doi.org/10.48550/arXiv.2202.05826}
\def \contactNAME{Sean McLeish}
\def \contactEMAIL{sean.mcleish@warwick.ac.uk}
\def \articleKEYWORDS{rescience c, Algorithmic Reasoning, Deep Thinking, Algorithm Synthesis, Python, machine learning, Deep Learning}
\def \journalNAME{ReScience C}
\def \journalVOLUME{9}
\def \journalISSUE{2}
\def \articleNUMBER{3}
\def \articleDOI{10.5281/zenodo.8173654}
\def \authorsFULL{Sean McLeish and Long Tran-Thanh}
\def \authorsABBRV{S. McLeish and L. Tran-Thanh}
\def \authorsSHORT{McLeish and Tran-Thanh}
\title{\articleTITLE}
\date{}
\author[1,\orcid{0000-0003-0206-3868}]{Sean McLeish}
\author[1,\orcid{0000-0003-1617-8316}]{Long Tran-Thanh}
\affil[1]{Department of Computer Science, University of Warwick, Coventry, England}
