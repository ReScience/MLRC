% DO NOT EDIT - automatically generated from metadata.yaml

\def \codeURL{https://github.com/kyosek/focus-reproducibility}
\def \codeDOI{10.5281/zenodo.7931344}
\def \codeSWH{swh:1:dir:e096a518285f9ee2f9ee2c5943293ba30f7e17b0}
\def \dataURL{https://github.com/a-lucic/focus}
\def \dataDOI{}
\def \editorNAME{Koustuv Sinha,\\ Maurits Bleeker,\\ Samarth Bhargav}
\def \editorORCID{}
\def \reviewerINAME{}
\def \reviewerIORCID{}
\def \reviewerIINAME{}
\def \reviewerIIORCID{}
\def \dateRECEIVED{04 February 2023}
\def \dateACCEPTED{19 April 2023}
\def \datePUBLISHED{20 July 2023}
\def \articleTITLE{[Re] FOCUS: Flexible Optimizable Counterfactual Explanations for Tree Ensembles}
\def \articleTYPE{Replication}
\def \articleDOMAIN{ML Reproducibility Challenge 2022}
\def \articleBIBLIOGRAPHY{bibliography.bib}
\def \articleYEAR{2023}
\def \reviewURL{https://openreview.net/forum?id=n1q-iz83S5&noteId=60kzDmcWau}
\def \articleABSTRACT{This study aims to reproduce the results of the paper FOCUS: Flexible Optimizable Counterfactual Explanations for Tree Ensembles by Lucic et al. The main claims of the original paper are that FOCUS is able to (i) generate counterfactual explanations for all the instances in a dataset; and (ii) find counterfactual explanations that are closer to the original input for tree-based algorithms than existing methods. This study replicates the original experiments using the code, data, and models provided by the authors. Additionally, this study re-implements code and retrains the models to evaluate the robustness and generality of FOCUS. This study was able to replicate the results of the original paper in terms of finding counterfactual explanations for all instances in datasets. Additional experiments were conducted to validate the robustness and generality of the conclusion. While there were slight deviations in terms of generating smaller mean distances, half of the models still outperformed the results of the existing method.}
\def \replicationCITE{Lucic, A., Oosterhuis, H., Haned, H., & de Rijke, M. (2019). "FOCUS Flexible Optimizable Counterfactual Explanations for Tree Ensembles." ArXiv. /abs/1911.12199}
\def \replicationBIB{lucic2022focus}
\def \replicationURL{https://arxiv.org/pdf/1911.12199.pdf}
\def \replicationDOI{https://doi.org/10.48550/arXiv.1911.12199}
\def \contactNAME{Kyosuke Morita}
\def \contactEMAIL{kyosuke1029@icloud.com}
\def \articleKEYWORDS{rescience c, python, machine learning, tensorflow, counterfactual explanations}
\def \journalNAME{ReScience C}
\def \journalVOLUME{9}
\def \journalISSUE{2}
\def \articleNUMBER{12}
\def \articleDOI{10.5281/zenodo.8173678}
\def \authorsFULL{Kyosuke Morita}
\def \authorsABBRV{K. Morita}
\def \authorsSHORT{Morita}
\title{\articleTITLE}
\date{}
\author[1,\orcid{0009-0008-3827-5700}]{Kyosuke Morita}
\affil[1]{Heidelberg University, Heidelberg, Germany}
