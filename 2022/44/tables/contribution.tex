\begin{table}[]
\centering
\footnotesize
\begin{tabular}{llllll}
\toprule
         & \textbf{Type}                       & \textbf{LSTM} & \textbf{T5} & \textbf{NQG} & \textbf{Neural-QCFG} \\
         \midrule
\textbf{SCAN}     & \multirow{2}{*}{Synthetic} & \texttt{Us}    &  \cite{shaw-etal-2021-compositional}  &   \cite{shaw-etal-2021-compositional}  &  \cite{kim2021sequencetosequence}           \\
\textbf{COGS}     &                            & \cite{kim-linzen-2020-cogs}     & \texttt{Us}  & -   & -           \\
\midrule
\textbf{GEOQUERY} & \multirow{2}{*}{Realistic} & \texttt{Us}    & \cite{shaw-etal-2021-compositional}   &  \cite{shaw-etal-2021-compositional}   & -           \\
\textbf{SPIDER}   &                            & \texttt{Us}    & \cite{shaw-etal-2021-compositional}   &  \cite{shaw-etal-2021-compositional}   & -           \\
\bottomrule
\end{tabular}
\caption{A list of datasets and models used in the papers.
%, a non-empty cell represents the model is fine-tuned/trained on the dataset. 
The type column indicates whether the dataset is synthetic or realistic. \texttt{Us} denotes additional experiments conducted by us.}
\label{tab:exp-list}
\end{table}