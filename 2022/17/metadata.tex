% DO NOT EDIT - automatically generated from metadata.yaml

\def \codeURL{https://github.com/CS-433/ml-project-2-ristrettolearning-2}
\def \codeDOI{}
\def \codeSWH{swh:1:dir:edcb01d0276ba2c09a8bd1fa035d6573d4d06fab}
\def \dataURL{}
\def \dataDOI{}
\def \editorNAME{Koustuv Sinha,\\ Maurits Bleeker,\\ Samarth Bhargav}
\def \editorORCID{}
\def \reviewerINAME{}
\def \reviewerIORCID{}
\def \reviewerIINAME{}
\def \reviewerIIORCID{}
\def \dateRECEIVED{04 February 2023}
\def \dateACCEPTED{19 April 2023}
\def \datePUBLISHED{20 July 2023}
\def \articleTITLE{[Re] Numerical influence of ReLU'(0) on backpropagation}
\def \articleTYPE{Replication}
\def \articleDOMAIN{ML Reproducibility Challenge 2022}
\def \articleBIBLIOGRAPHY{bibliography.bib}
\def \articleYEAR{2023}
\def \reviewURL{https://openreview.net/forum?id=YAWQTQZVoA}
\def \articleABSTRACT{Neural networks have become very common in machine learning, and new problems and trends arise as the trade-off between theory, computational tools and real-world problems become more narrow and complex. We decided to retake the influence of the ReLU'(0) on the backpropagation as it has become more common to use lower floating point precisions in the GPUs so that more tasks can run in parallel and make training and inference more efficient. As opposed to what theory suggests, the original authors shown that when using 16- and 32-bit precision, the value of ReLU'(0) may influence the result. In this work we extended some experiments to see how the training and test loss are affected in simple and more complex models.}
\def \replicationCITE{David Bertoin, Jérôme Bolte, Sébastien Gerchinovitz, Edouard Pauwels. "Numerical influence of ReLU'(0) on backpropagation" Advances in Neural Information Processing Systems. NeurIPS, 2021.}
\def \replicationBIB{Bertoin et al. "Numerical influence of ReLU'(0) on backpropagation"}
\def \replicationURL{https://proceedings.neurips.cc/paper_files/paper/2021/hash/043ab21fc5a1607b381ac3896176dac6-Abstract.html}
\def \replicationDOI{10.48550/arXiv.2106.12915}
\def \contactNAME{Tommaso Martorella}
\def \contactEMAIL{tommaso.martorella@epfl.ch}
\def \articleKEYWORDS{rescience c, machine learning, deep learning, python, pytorch}
\def \journalNAME{ReScience C}
\def \journalVOLUME{9}
\def \journalISSUE{2}
\def \articleNUMBER{17}
\def \articleDOI{}
\def \authorsFULL{Tommaso Martorella, Héctor Manuel Ramirez Contreras and Daniel Cerezo García}
\def \authorsABBRV{T. Martorella, H.M.R. Contreras and D.C. García}
\def \authorsSHORT{Martorella, Contreras and García}
\title{\articleTITLE}
\date{}
\author[1,\orcid{0000-0003-0136-4503}]{Tommaso Martorella}
\author[1,\orcid{0009-0005-4846-7712}]{Héctor Manuel Ramirez Contreras}
\author[1,\orcid{0009-0002-9755-6509}]{Daniel Cerezo García}
\affil[1]{École polytechnique fédérale de Lausanne (EPFL), Lausanne, Switzerland}
