% DO NOT EDIT - automatically generated from metadata.yaml

\def \codeURL{https://github.com/Mametchiii/lassi-reproducibility}
\def \codeDOI{10.5281/zenodo.7950717}
\def \codeSWH{swh:1:dir:ebe7321bfcc8268ca48b1b269c64b6fe1df79653}
\def \dataURL{}
\def \dataDOI{}
\def \editorNAME{Koustuv Sinha,\\ Maurits Bleeker,\\ Samarth Bhargav}
\def \editorORCID{}
\def \reviewerINAME{}
\def \reviewerIORCID{}
\def \reviewerIINAME{}
\def \reviewerIIORCID{}
\def \dateRECEIVED{04 February 2023}
\def \dateACCEPTED{19 April 2023}
\def \datePUBLISHED{20 July 2023}
\def \articleTITLE{[Re] Reproducibility Study of ”Latent Space Smoothing for Individually Fair Representations”}
\def \articleTYPE{Replication}
\def \articleDOMAIN{ML Reproducibility Challenge 2022}
\def \articleBIBLIOGRAPHY{bibliography.bib}
\def \articleYEAR{2023}
\def \reviewURL{https://openreview.net/forum?id=J-Lgb7Vc0wX}
\def \articleABSTRACT{Scope of Reproducibility: The aim of this work is to study the reproducibility of the paper 'Latent Space Smoothing for Individually Fair Representations' by Peychev et al., in which a novel representation learning method called LASSI is proposed. We aim to verify the three main claims made in the original paper: (1) LASSI increases certified individual fairness, while keeping prediction accuracies high, (2) LASSI can handle various sensitive attributes and attribute vectors and (3) LASSI representations can achieve high certified individual fairness even when downstream tasks are not known. In addition, we aim to test the robustness of their claims by conducting additional experiments. Methodology:  To reproduce the experiments, we use the step-by-step guidelines supplied by the original authors on their github repository. The experimental setup and datasets used are identical to the work reported in the original paper. We write additional code to run experiments beyond the scope of the work done by Peychev et al. In order to comply with resource limitations, we reproduce only the experiments relevant to the main claims. In total a budget of 45 hours on an NVIDIA Titan RTX GPU is used. Results: We are able to reproduce and verify the three main claims of the original paper, by reproducing the results within 5 percent of the reported values. The additional experiments were succesful and strengthen the claims that LASSI increases certified individual fairness compared to the baseline models. Outliers of the experiments are studied and found to be caused by biased and inaccurate input data. What was easy: Reproducing the original experiments was made possible by the extensive documentation and guidelines created by the authors in their code and public GitHub repository. The theoretical background provided in their paper was clear and detailed, allowing a deeper understanding about the inner workings of their models and metrics. What was difficult: The main difficulty was found within the complex structure of the original code files and the related functions across these files. The code needed to perform our additional experiments was therefore also complex and required us to alter many different functions in the original code.}
\def \replicationCITE{Momchil Peychev, Anian Ruoss, Mislav Balunovi and Maximilian Baader and Martin Vechev. “Latent Space Smoothing for Individually Fair Representations”}
\def \replicationBIB{peychev2022latent}
\def \replicationURL{https://arxiv.org/abs/2111.13650v3}
\def \replicationDOI{10.1007/978-3-031-19778-9_31}
\def \contactNAME{Didier Merk}
\def \contactEMAIL{didier.merk@gmail.com}
\def \articleKEYWORDS{machine learning, rescience c, rescience x, Latent space smoothing, LASSI, reproducibility, fairness, AI, deep learning, python, machine learning}
\def \journalNAME{ReScience C}
\def \journalVOLUME{9}
\def \journalISSUE{2}
\def \articleNUMBER{30}
\def \articleDOI{10.5281/zenodo.7950717}
\def \authorsFULL{Didier Merk et al.}
\def \authorsABBRV{D. Merk et al.}
\def \authorsSHORT{Merk et al.}
\title{\articleTITLE}
\date{}
\author[1,\orcid{0009-0005-0748-9018}]{Didier Merk}
\author[1,\orcid{0009-0008-0939-9018}]{Denny Smit}
\author[1,\orcid{0009-0005-9879-9205}]{Boaz Beukers}
\author[1,\orcid{0009-0007-0292-0213}]{Tsatsral Mendsuren}
\affil[1]{University of Amsterdam, Science Park, FNWI Department, Amsterdam, Netherlands}
