% DO NOT EDIT - automatically generated from metadata.yaml

\def \codeURL{https://github.com/seungjaeryanlee/pure-noise}
\def \codeDOI{10.5281/zenodo.7947264}
\def \codeSWH{swh:1:dir:41b1ddbd87720da65e78d56dfc86b8eb81dbba56}
\def \dataURL{}
\def \dataDOI{}
\def \editorNAME{Koustuv Sinha,\\ Maurits Bleeker,\\ Samarth Bhargav}
\def \editorORCID{}
\def \reviewerINAME{}
\def \reviewerIORCID{}
\def \reviewerIINAME{}
\def \reviewerIIORCID{}
\def \dateRECEIVED{04 February 2023}
\def \dateACCEPTED{19 April 2023}
\def \datePUBLISHED{20 July 2023}
\def \articleTITLE{[Re] Pure Noise to the Rescue of Insufficient Data}
\def \articleTYPE{Replication}
\def \articleDOMAIN{ML Reproducibility Challenge 2022}
\def \articleBIBLIOGRAPHY{bibliography.bib}
\def \articleYEAR{2023}
\def \reviewURL{https://openreview.net/forum?id=ErBe4MnsVD}
\def \articleABSTRACT{We examine the main claims of the original paper [1], which states that in an image classification task with imbalanced training data, (i) using pure noise to augment minority‐class images encourages generalization by improving minority-class accuracy. This method is paired with (ii) a new batch normalization layer that normalizes noise images using affine parameters learned from natural images, which improves the model’s performance. Moreover, (iii) this improvement is robust to varying levels of data augmentation. Finally, the authors propose that (iv) adding pure noise images can improve classification even on balanced training data.}
\def \replicationCITE{S. Zada, I. Benou, and M. Irani. “Pure Noise to the Rescue of Insufficient Data: Improving Imbalanced Classification by Training on Random Noise Images.” In: CoRR abs/2112.08810 (2021). arXiv: 2112.08810.}
\def \replicationBIB{PureNoise}
\def \replicationURL{https://arxiv.org/pdf/2112.08810.pdf}
\def \replicationDOI{}
\def \contactNAME{Seungjae Ryan Lee}
\def \contactEMAIL{ry@nlee.ai}
\def \articleKEYWORDS{rescience c, rescience x, python, pytorch, wandb, machine learning, deep learning}
\def \journalNAME{ReScience C}
\def \journalVOLUME{9}
\def \journalISSUE{2}
\def \articleNUMBER{45}
\def \articleDOI{10.5281/zenodo.8173763}
\def \authorsFULL{Seungjae Ryan Lee and Seungmin Brian Lee}
\def \authorsABBRV{S.R. Lee and S.B. Lee}
\def \authorsSHORT{Lee and Lee}
\title{\articleTITLE}
\date{}
\author[1,\orcid{0000-0002-4589-8892}]{Seungjae Ryan Lee}
\author[2,\orcid{0009-0006-7562-6886}]{Seungmin Brian Lee}
\affil[1]{Bloomberg LP, New York, NY, United States}
\affil[2]{Independent}
