% DO NOT EDIT - automatically generated from metadata.yaml

\def \codeURL{https://github.com/eomeragic1/g-mixup-reproducibility}
\def \codeDOI{}
\def \codeSWH{swh:1:dir:b3da6c4106727ef884219a1d04275a3306672c34}
\def \dataURL{}
\def \dataDOI{}
\def \editorNAME{}
\def \editorORCID{}
\def \reviewerINAME{}
\def \reviewerIORCID{}
\def \reviewerIINAME{}
\def \reviewerIIORCID{}
\def \dateRECEIVED{}
\def \dateACCEPTED{}
\def \datePUBLISHED{}
\def \articleTITLE{[Re] G-Mixup: Graph Data Augmentation for Graph Classification}
\def \articleTYPE{Replication}
\def \articleDOMAIN{ML Reproducibility Challenge 2022}
\def \articleBIBLIOGRAPHY{bibliography.bib}
\def \articleYEAR{2023}
\def \reviewURL{https://openreview.net/forum?id=XxUIomN-ndH}
\def \articleABSTRACT{G-Mixup is a novel data augmentation technique for graphs introduced by Han et al. We try to verify eight claims that the authors of G-Mixup make in their original paper by reproducing their experiments. The first two claims relate to the properties of graphons estimated from graphs, which are the main components of the method. Other claims relate to the superior performance of the method compared to other augmentation strategies.}
\def \replicationCITE{Han, Xiaotian, et al. "G-mixup: Graph data augmentation for graph classification." International Conference on Machine Learning. PMLR, 2022.}
\def \replicationBIB{han2022g}
\def \replicationURL{https://arxiv.org/pdf/2202.07179.pdf}
\def \replicationDOI{}
\def \contactNAME{Ermin Omeragić}
\def \contactEMAIL{eo3031@student.uni-lj.si}
\def \articleKEYWORDS{rescience c, machine learning, data augmentation, graph neural networks, pytorch geometric, mixup}
\def \journalNAME{ReScience C}
\def \journalVOLUME{9}
\def \journalISSUE{2}
\def \articleNUMBER{1}
\def \articleDOI{}
\def \authorsFULL{Ermin Omeragić and Vuk Đuranović}
\def \authorsABBRV{E. Omeragić and V. Đuranović}
\def \authorsSHORT{Omeragić and Đuranović}
\title{\articleTITLE}
\date{}
\author[1,\orcid{0009-0009-3960-522X}]{Ermin Omeragić}
\author[1,\orcid{0009-0002-5084-348X}]{Vuk Đuranović}
\affil[1]{University of Ljubljana, Faculty of Computer and Information Science, Večna pot 113, 1000 Ljubljana, Slovenia}
