% DO NOT EDIT - automatically generated from metadata.yaml

\def \codeURL{https://github.com/jichengyuan/MLRC2022_HIST}
\def \codeDOI{}
\def \codeSWH{swh:1:rev:aa9738a884343fb4798f8b06c7a8b9288eca6b14}
\def \dataURL{}
\def \dataDOI{}
\def \editorNAME{Koustuv Sinha,\\ Maurits Bleeker,\\ Samarth Bhargav}
\def \editorORCID{}
\def \reviewerINAME{}
\def \reviewerIORCID{}
\def \reviewerIINAME{}
\def \reviewerIIORCID{}
\def \dateRECEIVED{04 February 2023}
\def \dateACCEPTED{19 April 2023}
\def \datePUBLISHED{20 July 2023}
\def \articleTITLE{[Re] Hypergraph-Induced Semantic Tuplet Loss for Deep Metric Learning}
\def \articleTYPE{Replication}
\def \articleDOMAIN{ML Reproducibility Challenge 2022}
\def \articleBIBLIOGRAPHY{bibliography.bib}
\def \articleYEAR{2023}
\def \reviewURL{https://openreview.net/forum?id=JJQbk2hIQ5}
\def \articleABSTRACT{Our work reproduced the critical findings of the paper Hyper-graph-Induced Semantic Tuplet(HIST) Loss for Deep Metric Learning and investigate the effectiveness and robustness of HIST loss with the following five claims, which point that: (i) the proposed HIST loss performs consistently regardless of the batch size, (ii) regardless of the quantity of hyper-graph-neural-network(HGNN) layers L, the HIST loss shows consistent performance, (iii) the positive value of the scaling factor α of se‐ mantic tuplets brings reliable performance for modeling semantic relations of samples, (iv) the large temperature parameter τ is effective; if τ >16, HIST loss is insensitive to the scaling parameter. and (v) the HIST loss contributes to achieving SOTA performances under the standard evaluation settings.}
\def \replicationCITE{J. Lim, S. Yun, S. Park, and J. Y. Choi. "Hypergraph-induced semantic tuplet loss for deep metric learning." In:Proceedings of the IEEE/CVF Conference on Computer Vision and Pattern Recognition. 2022, pp. 212–222.}
\def \replicationBIB{lim2022hypergraph}
\def \replicationURL{https://openaccess.thecvf.com/content/CVPR2022/papers/Lim_Hypergraph-Induced_Semantic_Tuplet_Loss_for_Deep_Metric_Learning_CVPR_2022_paper.pdf}
\def \replicationDOI{10.1109/CVPR52688.2022.00031}
\def \contactNAME{Jicheng Yuan}
\def \contactEMAIL{jicheng.yuan@tu-berlin.de}
\def \articleKEYWORDS{rescience c, machine learning, deep learning, deep metric learning, hypergraph, semantic tuplet, python, pytorch}
\def \journalNAME{ReScience C}
\def \journalVOLUME{9}
\def \journalISSUE{2}
\def \articleNUMBER{8}
\def \articleDOI{10.5281/zenodo.8173666}
\def \authorsFULL{Jicheng Yuan and Danh Le-Phuoc}
\def \authorsABBRV{J. Yuan and D. Le-Phuoc}
\def \authorsSHORT{Yuan and Le-Phuoc}
\title{\articleTITLE}
\date{}
\author[1,\orcid{0009-0002-4448-2809}]{Jicheng Yuan}
\author[1,2,\orcid{0000-0003-2480-9261}]{Danh Le-Phuoc}
\affil[1]{Open Distributed Systems, Technical University of Berlin, Berlin, Germany}
\affil[2]{Fraunhofer Institute for Open Communication Systems, Berlin, Germany}
