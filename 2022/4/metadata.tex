% DO NOT EDIT - automatically generated from metadata.yaml

\def \codeURL{https://github.com/EricLangezaal/Re-Label-Free-XAI}
\def \codeDOI{https://doi.org/10.5281/zenodo.7885638}
\def \codeSWH{swh:1:dir:9a38c921b461cf9af72bc8f878d52da442fd39c1}
\def \dataURL{}
\def \dataDOI{}
\def \editorNAME{Koustuv Sinha,\\ Maurits Bleeker,\\ Samarth Bhargav}
\def \editorORCID{}
\def \reviewerINAME{}
\def \reviewerIORCID{}
\def \reviewerIINAME{}
\def \reviewerIIORCID{}
\def \dateRECEIVED{04 February 2023}
\def \dateACCEPTED{19 April 2023}
\def \datePUBLISHED{20 July 2023}
\def \articleTITLE{[Re] Label-Free Explainability for Unsupervised Models}
\def \articleTYPE{Replication}
\def \articleDOMAIN{ML Reproducibility Challenge 2022}
\def \articleBIBLIOGRAPHY{bibliography.bib}
\def \articleYEAR{2023}
\def \reviewURL{https://openreview.net/forum?id=bBVZ3pY4z8p}
\def \articleABSTRACT{Scope of Reproducibility — This study is an analysis of the reproducibility of the paper Label- Free Explainability for Unsupervised Models written by Jonathan Crabbé and Mihaela van der Schaar. The goal of this study is to verify the three main claims of the paper, which state that their new framework for label‐free explainability is capable of extending (i) linear feature importance as well as (ii) example importance methods to the unsupervised setting, whilst guaranteeing crucial properties, such as completeness and invariance with respect to latent symmetries. Finally, they use their framework to (iii) challenge some common beliefs about the interpretability of disentangled VAEs.
Methodology — The paper came with an extensive codebase containing all the necessary scripts to replicate the experiments in the paper. When comparing latent representations we also experimented with a different feature importance algorithm. Furthermore, we extended an experiment with the addition of a state‐of‐the‐art encoder.
Results — We find that the three main claims of the paper hold true as we were able to successfully reproduce each corresponding experiment. Our results differ in some minor aspects, but they do support the validity of the three main claims made in the paper. Furthermore, we also demonstrated that the framework proposed is expandable in new contexts, thus providing further support for its utility and applicability.
What was easy — The authors provided a substantial and thorough explanation within their appendix about the mathematical concepts of their work. Their repository was also supplemented with rigorous documentation which gave an excellent explanation of how to carry out various experiments.
What was difficult — Several experiments in the paper take multiple hours to execute, with one reproduction taking over 32 hours on the low‐cost computational setup used. Moreover, a few setup bugs were found, which meant that reproducing the experiments was a more strenuous task than simply executing a series of command‐line statements.
Communication with original authors — We initiated communication with the authors and inquired about specific experimental decisions made in their work. The authors provided a comprehensive response via email, addressing all questions raised.}
\def \replicationCITE{Label-Free Explainability for Unsupervised Models}
\def \replicationBIB{LabelFreeExplainability}
\def \replicationURL{https://arxiv.org/pdf/2203.01928}
\def \replicationDOI{https://doi.org/10.48550/arXiv.2203.01928}
\def \contactNAME{Eric Robin Langezaal}
\def \contactEMAIL{eric.langezaal@gmail.com}
\def \articleKEYWORDS{rescience c, machine learning, Reproducibility, label-free, unsupervised learning, explainable Artificial Intelligence, feature importance, example importance, disentangled VAE}
\def \journalNAME{ReScience C}
\def \journalVOLUME{9}
\def \journalISSUE{2}
\def \articleNUMBER{4}
\def \articleDOI{10.5281/zenodo.8173656}
\def \authorsFULL{Eric Robin Langezaal et al.}
\def \authorsABBRV{E.R. Langezaal et al.}
\def \authorsSHORT{Langezaal et al.}
\title{\articleTITLE}
\date{}
\author[1,2,\orcid{0009-0008-5614-1749}]{Eric Robin Langezaal}
\author[1,2,\orcid{0009-0006-3580-7535}]{Jesse Belleman}
\author[1,2,\orcid{0009-0004-5883-9500}]{Tim Veenboer}
\author[1,2,\orcid{0009-0002-1243-5346}]{Joeri Noorthoek}
\affil[1]{Equal contributions}
\affil[2]{FNWI, University of Amsterdam, Amsterdam, The Netherlands}
