% DO NOT EDIT - automatically generated from metadata.yaml

\def \codeURL{https://github.com/jwagenbach/FACT/}
\def \codeDOI{10.5281/zenodo.7895731}
\def \codeSWH{swh:1:dir:0df843f2fc8c0868968429afb8908db7d3a76a3c}
\def \dataURL{}
\def \dataDOI{}
\def \editorNAME{Koustuv Sinha,\\ Maurits Bleeker,\\ Samarth Bhargav}
\def \editorORCID{}
\def \reviewerINAME{}
\def \reviewerIORCID{}
\def \reviewerIINAME{}
\def \reviewerIIORCID{}
\def \dateRECEIVED{04 February 2023}
\def \dateACCEPTED{19 April 2023}
\def \datePUBLISHED{20 July 2023}
\def \articleTITLE{[Re] Reproducibility study of ”Label-Free Explainability for Unsupervised Models”}
\def \articleTYPE{Replication}
\def \articleDOMAIN{ML Reproducibility Challenge 2022}
\def \articleBIBLIOGRAPHY{bibliography.bib}
\def \articleYEAR{2023}
\def \reviewURL{https://openreview.net/forum?id=n2qXFXiMsAM}
\def \articleABSTRACT{In this work, we present our reproducibility study of 'Label-Free Explainability for Unsupervised Models', a paper that introduces two post‐hoc explanation techniques for neural networks: (1) label‐free feature importance and (2) label‐free example importance. Our study focuses on the reproducibility of the authors' most important claims: (i) perturbing features with the highest importance scores causes higherlatent shift than perturbing random pixels, (ii) label‐free example importance scores help to identify training examples that are highly related to a given test example, (iii) unsupervised models trained on different tasks show moderate correlation among the highest scored features and (iv) low correlation in example scores measured on a fixed set of data points, and (v) increasing the disentanglement with β in a β‐VAE does not imply that latent units will focus on more different features. We reviewed the authors' code, checked if the implementation of experiments matched with the paper, and also ran all experiments. The results are shown to be reproducible. Moreover, we extended the codebase in order to run the experiments on more datasets, and to test the claims with other experiments.}
\def \replicationCITE{Crabbé et al., 2022}
\def \replicationBIB{crabbe2022label}
\def \replicationURL{https://proceedings.mlr.press/v162/crabbe22a/crabbe22a.pdf}
\def \replicationDOI{}
\def \contactNAME{Gergely Papp}
\def \contactEMAIL{gergely.papp@student.uva.nl}
\def \articleKEYWORDS{rescience c, machine learning, reproduction, explainability, unlabeled, representations, python, torch}
\def \journalNAME{ReScience C}
\def \journalVOLUME{9}
\def \journalISSUE{2}
\def \articleNUMBER{24}
\def \articleDOI{}
\def \authorsFULL{Gergely Papp et al.}
\def \authorsABBRV{G. Papp et al.}
\def \authorsSHORT{Papp et al.}
\title{\articleTITLE}
\date{}
\author[1,2,\orcid{0009-0007-0454-9788}]{Gergely Papp}
\author[1,\orcid{0009-0000-9171-8947}]{Julius Wagenbach}
\author[1,\orcid{0009-0009-5314-2065}]{Laurens Jans de Vries}
\author[1,\orcid{0000-0002-9136-1498}]{Niklas Mather}
\affil[1]{University of Amsterdam, Amsterdam, the Netherlands}
\affil[2]{Alfréd Rényi Institute of Mathematics, Budapest, Hungary}
