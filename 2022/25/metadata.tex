% DO NOT EDIT - automatically generated from metadata.yaml

\def \codeURL{https://github.com/ErikBuis/FACT2023}
\def \codeDOI{https://zenodo.org/badge/latestdoi/585223762}
\def \codeSWH{swh:1:dir:3b9cb41cd6d8680801f3b80b29410b641408348e}
\def \dataURL{}
\def \dataDOI{}
\def \editorNAME{Koustuv Sinha,\\ Maurits Bleeker,\\ Samarth Bhargav}
\def \editorORCID{}
\def \reviewerINAME{}
\def \reviewerIORCID{}
\def \reviewerIINAME{}
\def \reviewerIIORCID{}
\def \dateRECEIVED{04 February 2023}
\def \dateACCEPTED{19 April 2023}
\def \datePUBLISHED{20 July 2023}
\def \articleTITLE{[Re] Reproducibility study of ``Explaining Deep Convolutional Neural Networks via Latent Visual-Semantic Filter Attention''}
\def \articleTYPE{Replication}
\def \articleDOMAIN{ML Reproducibility Challenge 2022}
\def \articleBIBLIOGRAPHY{bibliography.bib}
\def \articleYEAR{2023}
\def \reviewURL{https://openreview.net/forum?id=nsrHznwHhl}
\def \articleABSTRACT{Scope of Reproducibility In this work, we aim to reproduce the findings of the paper Explaining Deep Convolutional Neural Networks via Latent Visual-Semantic Filter Attention (LaViSE). This paper presents a global post-hoc explanation framework for deep learning models that generates semantic explanations for CNN filters. To assess the reproducibility of this work, we verify the main claims made in the paper. More specifically, we evaluate whether the framework creates an accurate mapping to the semantic space, generates words which were not seen in the training data, and is able to generalize to any pre-trained CNN.
Methodology To reproduce the experiments detailed in the original paper, we first obtained the author's code. However, we had to modify the code for the experiments to be executable, adding missing code, debugging, and making the code more maintainable. Additionally, we evaluated the model's generalizability to other CNNs. The project required a total of 62 GPU hours.
Results Our recall scores and qualitative experiments validate all claims of the authors: the framework creates an accurate mapping between the visual and semantic space, can analyze any trained CNN regardless of original training data availability, and is able to generate novel out-of-dataset descriptions for filters.
What was easy The paper was well-written and easy to understand, with helpful figures illustrating the LaViSE framework that aided in the implementation process.
What was difficult The implementation of the methodology outlined in the paper was particularly challenging due to limited documentation and insufficient details about parts that were not implemented in the existing codebase. Additionally, some experiments could not be recreated because they would require a significant amount of resources to verify.
Communication with original authors We contacted the authors to clarify missing information and aspects that were not functioning as expected. However, we did not receive a response to our questions.}
\def \replicationCITE{Y. Yang, S. Kim and J. Joo, "Explaining Deep Convolutional Neural Networks via Latent Visual-Semantic Filter Attention," 2022 IEEE/CVF Conference on Computer Vision and Pattern Recognition (CVPR), New Orleans, LA, USA, 2022, pp. 8323-8333, doi: 10.1109/CVPR52688.2022.00815.}
\def \replicationBIB{yang2022explaining}
\def \replicationURL{https://arxiv.org/pdf/2204.04601.pdf}
\def \replicationDOI{10.1109/CVPR52688.2022.00815}
\def \contactNAME{Erik Buis}
\def \contactEMAIL{ebbuis@gmail.com}
\def \articleKEYWORDS{Reproduce, Interpretability, Convolutional Neural Networks, Post-hoc global method, Latent Visual Semantic Filter Attention, Python, PyTorch, Common Objects in Context, machine learning, rescience c}
\def \journalNAME{ReScience C}
\def \journalVOLUME{9}
\def \journalISSUE{2}
\def \articleNUMBER{25}
\def \articleDOI{}
\def \authorsFULL{Erik Buis, Sebastiaan Dijkstra and Bram Heijermans}
\def \authorsABBRV{E. Buis, S. Dijkstra and B. Heijermans}
\def \authorsSHORT{Buis, Dijkstra and Heijermans}
\title{\articleTITLE}
\date{}
\author[1,\orcid{0009-0000-9149-4824}]{Erik Buis}
\author[1,\orcid{0009-0002-6402-5816}]{Sebastiaan Dijkstra}
\author[1,\orcid{0009-0005-2623-8396}]{Bram Heijermans}
\affil[1]{University of Amsterdam, Amsterdam, The Netherlands}
