% DO NOT EDIT - automatically generated from metadata.yaml

\def \codeURL{https://github.com/vicliv/AMuLaP-Reproduction}
\def \codeDOI{}
\def \codeSWH{swh:1:dir:44eec5466ce396c0e4af4cba3e28b994c4a82600}
\def \dataURL{}
\def \dataDOI{}
\def \editorNAME{Koustuv Sinha,\\ Maurits Bleeker,\\ Samarth Bhargav}
\def \editorORCID{}
\def \reviewerINAME{}
\def \reviewerIORCID{}
\def \reviewerIINAME{}
\def \reviewerIIORCID{}
\def \dateRECEIVED{04 February 2023}
\def \dateACCEPTED{19 April 2023}
\def \datePUBLISHED{20 July 2023}
\def \articleTITLE{[Re] A Reproduction of Automatic Multi-Label Prompting: Simple and Interpretable Few-Shot Classification}
\def \articleTYPE{Replication}
\def \articleDOMAIN{ML Reproducibility Challenge 2022}
\def \articleBIBLIOGRAPHY{bibliography.bib}
\def \articleYEAR{2023}
\def \reviewURL{https://openreview.net/forum?id=t8ZZ2Y356Ix}
\def \articleABSTRACT{Prompt-based learning with pre-trained large language models (PLM) have recently gained popularity and much work has been done in searching for prompting templates that best allow the PLM to treat tasks as masked language modeling problems. But when we want the PLM to fill in masks from a finite set of label words, finding good label word sets is just as important. In this work, we reproduce and extend the results of the work Automatic Multi-Label Prompting: Simple and Interpretable Few-Shot Classification in which they proposed a novel technique for automatic multi-label map search through prompting. We use the original codebase to verify the originally reported results, and extend experiments on 2 new “real-world” datasets. We reproduced the original paper's metrics within the standard deviation reported, verifying the paper's claims that its proposed method achieves competitive results and beats some baselines under different settings (whether or not models are fine-tuned). Our extended experiments show the method's potential in being used on real-world data, but uncovers new considerations regarding prompt template, language model choice, and seed choice to achieve satisfactory performance.}
\def \replicationCITE{H. Wang, C. Xu, and J. McAuley. “Automatic Multi-Label Prompting: Simple and Interpretable Few-Shot Classification.” In: Proceedings of the 2022 Conference of the North American Chapter of the Association for Computational Linguistics: Human Language Technologies. Seattle, United States: Association for Computational Linguistics, July 2022, pp. 5483-5492}
\def \replicationBIB{Wang}
\def \replicationURL{https://aclanthology.org/2022.naacl-main.401.pdf}
\def \replicationDOI{10.18653/v1/2022.naacl-main.401}
\def \contactNAME{Victor Livernoche}
\def \contactEMAIL{victor.livernoche@mail.mcgill.ca}
\def \articleKEYWORDS{rescience c, rescience x, python, machine learning, transformers, prompt engineering, natural language processing, multi-label prompting, few-shot classification}
\def \journalNAME{ReScience C}
\def \journalVOLUME{9}
\def \journalISSUE{2}
\def \articleNUMBER{33}
\def \articleDOI{}
\def \authorsFULL{Victor Livernoche and Vidya Sujaya}
\def \authorsABBRV{V. Livernoche and V. Sujaya}
\def \authorsSHORT{Livernoche and Sujaya}
\title{\articleTITLE}
\date{}
\author[1,2,\orcid{0000-0002-3819-730X}]{Victor Livernoche}
\author[1,\orcid{0009-0007-7079-7624}]{Vidya Sujaya}
\affil[1]{McGill University, Montréal, Qc, Canada}
\affil[2]{Mila - Québec AI Institute, Montréal, Qc, Canada}
