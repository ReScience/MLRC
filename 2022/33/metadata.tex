\def \codeURL{https://github.com/vicliv/AMuLaP-Reproduction}
\def \codeDOI{}
\def \dataURL{}
\def \dataDOI{}
\def \editorNAME{}
\def \editorORCID{}
\def \reviewerINAME{}
\def \reviewerIORCID{}
\def \reviewerIINAME{}
\def \reviewerIIORCID{}
\def \dateRECEIVED{}
\def \dateACCEPTED{}
\def \datePUBLISHED{}
\def \articleTITLE{A Reproduction of Automatic Multi-Label Prompting: Simple and Interpretable Few-Shot Classification}
\def \articleTYPE{Replication / ML Reproducibility Challenge 2022}
\def \articleDOMAIN{}
\def \articleBIBLIOGRAPHY{bibliography.bib}
\def \articleYEAR{2022}
\def \reviewURL{}
\def \articleABSTRACT{
In this paper, we will reproduce and extend the results of an article released at the NAACL 2022 conference called Automatic Multi-Label Prompting: Simple and Interpretable Few-Shot Classification \cite{Wang}. To do this, we will use the code released by the authors and improve its modularity to ensure this method is more available to others. The paper showed a method called AMuLaP to generate multiple labels for a classification task using prompting on a pretrained large language model. This method is useful since a good mapping from the original task labels to tokens is critical to improving few‐shot performance on text classification tasks. Given time and computational limitations, we will prioritize reproducing the results AMuLaP on the Stanford Sentiment Treebank (SST‐2) and Multi‐Genre Natural Language Inference (MNLI) datasets. They are the most similar to the dataset we want to create as we want to evaluate AMuLaP for real-world scenarios. We will then extend on two new datasets: Scale AI and Oxford’s DEBAGREEMENT dataset and manually compiled data of Donald Trump’s tweets, classified into ”flagged” and ”safe”. Furthermore, we will try the method on a different language model, BLOOM \cite{BLOOM}, to see if it can create more nuanced labels and find out which prompt template works best with AMuLaP. 
}
\def \replicationCITE{}
\def \replicationBIB{H. Wang, C. Xu, and J. McAuley. Automatic Multi-Label Prompting: Simple and Interpretable Few-Shot Classification. 2022}
\def \replicationURL{https://arxiv.org/abs/2204.06305}
\def \replicationDOI{}
\def \contactNAME{Victor Livernoche}
\def \contactEMAIL{victor.livernoche@mail.mcgill.ca}
\def \articleKEYWORDS{rescience c, rescience x}
\def \journalNAME{ReScience C}
\def \journalVOLUME{9}
\def \journalISSUE{2}
\def \articleNUMBER{}
\def \articleDOI{}
\def \authorsFULL{Vidya Sujaya and Victor Livernoche}
\def \authorsABBRV{V. Livernoche, V. Sujaya}
\def \authorsSHORT{V. Livernoche, V. Sujaya}
\title{\articleTITLE}
\date{}
\author[1,\orcid{0000-0002-3819-730X}]{Victor Livernoche}
\author[1,\orcid{0000-0000-0000-0000}]{Vidya Sujaya}
\affil[1]{McGill University}
