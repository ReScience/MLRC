% DO NOT EDIT - automatically generated from metadata.yaml

\def \codeURL{https://github.com/matteobrv/repro-homonymy-acl21}
\def \codeDOI{10.5281/zenodo.7886291}
\def \codeSWH{swh:1:dir:c0cbb81bf15ace80d98da9d95fed3440046539f9}
\def \dataURL{}
\def \dataDOI{}
\def \editorNAME{}
\def \editorORCID{}
\def \reviewerINAME{}
\def \reviewerIORCID{}
\def \reviewerIINAME{}
\def \reviewerIIORCID{}
\def \dateRECEIVED{}
\def \dateACCEPTED{}
\def \datePUBLISHED{}
\def \articleTITLE{[Re] Exploring the Representation of Word Meanings in Context}
\def \articleTYPE{Replication}
\def \articleDOMAIN{ML Reproducibility Challenge 2022}
\def \articleBIBLIOGRAPHY{bibliography.bib}
\def \articleYEAR{2023}
\def \reviewURL{https://openreview.net/forum?id=Od5dD58libt}
\def \articleABSTRACT{This report summarizes our efforts to reproduce the results presented in the ACL2021 paper Exploring the Representation of Word Meanings in Context: A Case Study on Homonymy and Synonymy by Marcos Garcia, as part of the ML Reproducibility Challenge 2022. The original author looks at both static and contextualized word embeddings to assess their ability to adequately represent different lexical-semantic relations, such as homonymy and synonymy. While the original paper describes experiments carried out on a number of contextualized and static models, we limit our reproducibility attempt to the results reported for BERT and fastText. Moreover, we extend the original work by compiling a new Italian dataset and report our findings for this additional resource. Despite being able to reproduce the original results only in part, the hypotheses formulated in the original paper are still corroborated.}
\def \replicationCITE{Garcia (2021). Exploring the Representation of Word Meanings in Context. A Case Study on Homonymy and Synonymy (ACL 2021).}
\def \replicationBIB{garcia:acl}
\def \replicationURL{https://aclanthology.org/2021.acl-long.281/}
\def \replicationDOI{10.18653/v1/2021.acl-long.281}
\def \contactNAME{Çağrı Çöltekin}
\def \contactEMAIL{ccoltekin@sfs.uni-tuebingen.de}
\def \articleKEYWORDS{rescience c, machine learning, deep learning, NLP, fasttext, BERT}
\def \journalNAME{ReScience C}
\def \journalVOLUME{9}
\def \journalISSUE{2}
\def \articleNUMBER{}
\def \articleDOI{}
\def \authorsFULL{Matteo Brivio and Çağrı Çöltekin}
\def \authorsABBRV{M. Brivio and Ç. Çöltekin}
\def \authorsSHORT{Brivio and Çöltekin}
\title{\articleTITLE}
\date{}
\author[1,\orcid{0000-0001-6273-6900}]{Matteo Brivio}
\author[1,\orcid{0000-0003-1031-6327}]{Çağrı Çöltekin}
\affil[1]{University of Tübingen, Tübingen, Germany}
