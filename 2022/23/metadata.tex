% DO NOT EDIT - automatically generated from metadata.yaml

\def \codeURL{https://github.com/WalterSimoncini/CGD-Reproduction}
\def \codeDOI{10.5281/zenodo.7998663}
\def \codeSWH{swh:1:dir:4a89288fc050158c419caee05af572cad7b71a12}
\def \dataURL{}
\def \dataDOI{}
\def \editorNAME{Koustuv Sinha,\\ Maurits Bleeker,\\ Samarth Bhargav}
\def \editorORCID{}
\def \reviewerINAME{}
\def \reviewerIORCID{}
\def \reviewerIINAME{}
\def \reviewerIIORCID{}
\def \dateRECEIVED{04 February 2023}
\def \dateACCEPTED{19 April 2023}
\def \datePUBLISHED{20 July 2023}
\def \articleTITLE{[Re] Reproducibility Study of ”Focus On The Common Good: Group Distributional Robustness Follows”}
\def \articleTYPE{Replication}
\def \articleDOMAIN{ML Reproducibility Challenge 2022}
\def \articleBIBLIOGRAPHY{bibliography.bib}
\def \articleYEAR{2023}
\def \reviewURL{https://openreview.net/forum?id=ye8PftiQLQ}
\def \articleABSTRACT{- Scope of Reproducibility This paper attempts to reproduce the main claims of Focus On The Common Good: Group Distributional Robustness Follows by Piratla et al., which introduces Common Gradient Descent (CGD), a novel optimization algorithm for handling spurious correlations and sub-population shifts. We have identified three central claims: (I) CGD is more robust than Group-DRO and leads to the largest average loss decrease across all groups (II) CGD generalizes better across all groups in comparison to ERM, and (III) CGD monotonically decreases the group-average loss. - Methodology The experiments of this paper are based on the open source implementation of CGD released by the authors, which required some modifications to work with the latest version of the WILDS framework. - Results The results from our experiments were overall in line with the paper. We show that CGD outperforms Group-DRO on synthetic datasets with induced spurious correlations, but the benefits of CGD are not clear in a real-world setting. Beyond the results of the original paper, our attempt to empirically verify the mathematical proof of the authors that CGD monotonically decreases the loss was not conclusive. - What was easy The implementation from the original paper was available on GitHub with detailed instructions provided in the documentation. It was also relatively easy to introduce additional datasets and algorithms to the WILDS codebase. - What was difficult The CGD implementation and several experiments could not be run out-of-the-box and required major modifications to run with the latest version of WILDS. The majority of the hyperparameter values for the experiments were not clearly stated. Lastly, the experiments were computationally expensive and required 440 GPU hours. - Communication with original authors We reached out to the original authors to request additional information about the hyperparameter values and the implementation of some experiments. The authors promptly responded with sources for the hyperparameters, useful information about WILDS and provided some missing parts of the code. Overall, the communications were timely and effective.}
\def \replicationCITE{V. Piratla, P. Netrapalli, and S. Sarawagi. “Focus on the Common Good: Group Distributional Robustness Follows.” In: Proceedings of the International Conference on Learning Representations (2022).}
\def \replicationBIB{piratla2022focus}
\def \replicationURL{https://openreview.net/pdf?id=irARV_2VFs4}
\def \replicationDOI{10.48550/arXiv.2110.02619}
\def \contactNAME{Walter Simoncini}
\def \contactEMAIL{walter.simoncini@student.uva.nl}
\def \articleKEYWORDS{rescience c, machine learning, deep learning, python, pytorch, robustness}
\def \journalNAME{ReScience C}
\def \journalVOLUME{9}
\def \journalISSUE{2}
\def \articleNUMBER{23}
\def \articleDOI{10.5281/zenodo.7998663}
\def \authorsFULL{Walter Simoncini et al.}
\def \authorsABBRV{W. Simoncini et al.}
\def \authorsSHORT{Simoncini et al.}
\title{\articleTITLE}
\date{}
\author[1,\orcid{0009-0006-3086-7141}]{Walter Simoncini}
\author[1,\orcid{0000-0002-9223-9726}]{Ioanna Gogou}
\author[1,\orcid{0009-0007-9039-1349}]{Marta Freixo Lopes}
\author[1,\orcid{0009-0005-8973-7133}]{Ron Kremer}
\affil[1]{University of Amsterdam, Amsterdam, The Netherlands}
