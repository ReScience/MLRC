\section{Scope of reproducibility}
\label{sec:claims}

\jdcomment{Explain the claims from the original paper you picked for the reproduction study and briefly motivate your choice. Try to summarize each claim  in 1-2 sentences, e.g. "The introduced activation function X outperforms a similar activation function Y on tasks Z,V,W". Make the scope as specific as possible. It should be something that can be supported or rejected by your data. For example, this scope is too broad and lacks precise outcome: "Contextual embedding models have shown strong performance on a number of tasks across NLP. We will run experiments evaluating two types of contextual embedding models on datasets X, Y, and Z." This scope is better because it's more specific and has an outcome that can be either supported or rejected based on your results: "Finetuning pretrained BERT on SST-2 will have higher accuracy than an LSTM trained with GloVe embeddings."}

In order to verify the central claims presented in the paper we focus on the following target questions:

\begin{itemize}
    \item Does \textit{RigL} outperform existing sparse-to-sparse training techniques---such as SET (\citet{Mocanu2018SET}) and SNFS (\citet{dettmers2020sparse})---and match the accuracy of dense-to-sparse training methods such as iterative pruning (\citet{to_prune_or_not})?
    
    \item \textit{RigL} requires two additional hyperparameters to tune. We investigate the sensitivity of final performance to these hyperparameters across a variety of target sparsities (Section \ref{hyperparameter-tuning}).
    
    \item How does the choice of sparsity initialization affect the final performance for a fixed parameter count and a fixed training budget (Section \ref{effect-sparsity-distribution})?
    
    \item Does redistributing layer-wise sparsity during connection updates (\citet{dettmers2020sparse}) improve \textit{RigL}'s performance? Can the final layer-wise distribution serve as a good sparsity initialization scheme (Section \ref{effect-redistribution})? 

\end{itemize}

\jdcomment{Each experiment in Section~\ref{sec:results} will support (at least) one of these claims, so a reader of your report should be able to understand the \emph{arguments} (claims) and, separately, the \emph{evidence} that supports them.}
%\jdcomment{To organizers: I asked my students to connect the main claims and the experiments that supported them. For example, in this list above they could have ``Claim 1, which is supported by Experiment 1 in Figure 1.'' The benefit was that this caused the students to think about what their experiments were showing (as opposed to blindly rerunning each experiment and not considering how it fit into the overall story), but honestly it seemed hard for the students to understand what I was asking for.}