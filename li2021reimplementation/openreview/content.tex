
\begin{abstract}
FixMatch is a semi-supervised learning method, which achieves comparable results with fully supervised learning by leveraging a limited number of labeled data (pseudo labelling technique) and taking a good use of the unlabeled data (consistency regularization ). 
In this work, we reimplement FixMatch and achieve reasonably comparable performance with the official implementation, which supports that FixMatch outperforms semi-superivesed learning benchmarks and demonstrates that the author's choices with respect to those ablations were experimentally sound.
 Next, we investigate the existence of a major problem of FixMatch, \textit{confirmation errors}, by reconstructing the batch structure during the training process. It reveals existing confirmation errors, especially the ones caused by \textit{asymmetric noise} in pseudo labels. 
To deal with the problem, we apply equal-frequency and confidence entropy regularization to the labeled data and add them in the loss function. Our experimental results on CIFAR-10 show that using either of the entropy regularization in the loss function can reduce the asymmetric noise in pseudo labels and improve the performance of FixMatch in the presence of (pseudo) labels containing (asymmetric) noise. Our code is available at the url: \url{https://github.com/Celiali/FixMatch}.
\end{abstract}

% \input{essay}
\section{Introduction}

Most prominent vision datasets are afflicted by \emph{contextual bias}. For example, ``microwave" typically is found in kitchens, which also contain objects like ``refrigerator" and ``oven."  Such co-occurrence patterns may inadvertently induce contextual bias in datasets, which could consequently seep into models trained on them. When models overly rely on context, they may not generalize to settings where typical co-occurrence patterns are absent. The original paper by Singh et al.~\cite{Singh_2020_CVPR} proposes two methods for mitigating such contextual biases and improving the robustness of the learnt feature representations. The paper demonstrates their methods on multi-label object and attribute classification tasks, using the COCO-Stuff~\cite{caesar2018cvpr}, DeepFashion~\cite{liuLQWTcvpr16DeepFashion}, Animals with Attributes (AwA)~\cite{AwA}, and UnRel~\cite{Peyre17} datasets. Our exploration centers on four main directions:\\
\\
First, we trained the baseline classifier presented in the paper (Section~\ref{sec:baselineimplementation} for implementation and training details; Sections~\ref{sec:evaluation}-\ref{sec:baselineresults} for results). Due to likely implementation discrepancies, our results differed from the original paper by 0.6--3.1\% mAP on COCO-Stuff, by 0.7--1.4\% top-3 recall on DeepFashion, and by 0.1--3.2\% mAP on AwA (Table~\ref{tab:mainresults}). We ran a hyperparameter search (Appendix~\ref{sec:hyperparametersearch}), which yielded a significant (1.4--3.6\%) improvement on DeepFashion.\\
\\
Next, we identified the \emph{biased categories} in each dataset, i.e., visual categories that suffer from contextual bias. We followed the proposed method of using the baseline classifier to identify these categories, and discovered that the classifier implementation has a non-trivial effect. For COCO-Stuff, 18 of the top-20 categories we identified matched the original paper's top-20 categories (10 on DeepFashion, 18 on AwA; Section~\ref{sec:biasedcategories}). Nevertheless, the categories we identified appear reasonable  (e.g., ``fork" co-occurs with ``dining table"; Appendix~\ref{sec:biasedcategoriesapp}). As training and evaluation of most methods depend on the biased categories, we used the paper's biased categories for subsequent experiments.\\
\\
Third, we checked the main claim of the paper, that the proposed \emph{CAM-based} and \emph{feature-split} methods help improve recognition of biased categories in the absence of their context (Section~\ref{sec:stage2}). On COCO-Stuff, DeepFashion, and UnRel, we were able to reproduce the improvements gained from the proposed \textit{feature-split} method towards reducing contextual bias, whereas on AwA, we saw a drop in performance. The proposed \textit{CAM-based} method, which was only applied to COCO-Stuff, also helped reduce contextual bias, though not as significantly as the \textit{feature-split} method. For the  method, we reproduced the original paper's results to within 0.5\% mAP (Section~\ref{sec:mainresults}). We also successfully reproduced the paper's weight similarity analysis, as well as the qualitative analyses with class activation maps (CAMs)~\cite{zhou2015cnnlocalization}.\\
\\
Lastly, we ran additional experiments and ablation studies (Section~\ref{sec:addanalyses}). These revealed that the regularization term in the \emph{CAM-based} method and the weighted loss in the \emph{feature-split} method are central to the methods' performance. We also observed that varying the feature subspace size influences the \emph{feature-split} method's accuracy.

\input{../openreview/related work.tex}
%%% METHODOLOGY %%%
\section{Methodology} \label{sec:methods}

Professor Vassilis Kalantzis, who we contacted via email, generously provided us with the relevant MATLAB code and data; however, we chose to re-implement the algorithm from scratch in Python with standard packages (NumPy \cite{numpy}, SciPy \cite{scipy}, and scikit-learn \cite{scikit-learn}) and used the MATLAB code to confirm our implementation.
We compared the performance of Algorithms 2.1 and 2.2 with \verb|FrequentDirections| \cite{Ghashami2016}, a state-of-the-art streaming algorithm.
Experiments were conducted on a MacBook Pro with a 2.3 GHz Dual-Core Intel Core i5 processor with 16 GB of RAM, and the code is publicly available on GitHub\footnote{https://anonymous.4open.science/r/truncatedSVD-0162/}.
All plots were generated using Matplotlib \cite{matplotlib}.

%% IMPLEMENTATION
\subsection{Implementation}

We chose to implement the three truncated SVD update algorithms as methods of an \verb|EvolvingMatrix| class, which we will refer to as \verb|EM| from here on out.
With each experiment, the \verb|EM| class was initialized with various parameters (initial matrix, matrix to be appended, number of batches, etc.) and updates were carried out using one of the update methods.
A simplified version of the experiment is shown in Listing \ref{lst:exp_pseudo}.
Algorithms 2.1 and 2.2 were written based on the pseudo-code presented in Algorithm \ref{alg:row-update}, where the $Z$ and $Z^H A$ matrices were calculated using their respective formulas. 

\lstinputlisting[language=python, float, floatplacement=h, caption=Simplified experiment structure, label={lst:exp_pseudo}]{../openreview/sections/exp.py}

\paragraph{Algorithm 2.1}

The $Z$ and $Z^H A$ matrices were constructed as in Equations \ref{eq:zha-simon_Z} and \ref{eq:zha-simon_ZHA}, respectively. 

\paragraph{Algorithm 2.2}

The main difficulty in implementing Algorithm 2.2 was in the calculation of $X_{\lambda,r}$.
We chose to solve for $X$ in Equation \ref{eq:bcg} using the block Conjugate Gradient method (BCG) \cite{OLeary1980} as recommended in \cite{Kalantzis2021}.
Though \cite{Kalantzis2021} specified, at maximum, one iteration of BCG, we found that the MATLAB code set the limit to two iterations.
As the additional iteration did not greatly increase the computational cost, we chose to run BCG a maximum of two iterations as well.
Once $X$ was calculated, we calculated $X_{\lambda,r}$ as per Equation \ref{eq:bcg_rsvd} using randomized SVD \cite{Halko2011}.
For this, we used the scikit-learn \verb|randomized_svd| implementation \cite{scikit-learn}.
Based on the description for calculating $X_{\lambda,r}$ in \cite{Kalantzis2021}, we set \verb|n_components|$=r$, \verb|n_oversamples|$=2r$, and \verb|n_iter|$=0$.
The $X_{\lambda,r}$ returned was then used to calculate $Z$ and $Z^H A$ as in Equations \ref{eq:enhanced_Z} and \ref{eq:enhanced_ZHA}, respectively.

\paragraph{Frequent Directions}

A modified version of \verb|FrequentDirections|\footnote{https://github.com/edoliberty/frequent-directions} was incorporated as an update method into the $\verb|EM|$ class.
Since \verb|FrequentDirections| is a line-by-line update method as opposed to a batch update method, the update method in the $\verb|EM|$ class was constructed to receive a matrix $E$ containing the rows to be added and performs the \verb|FrequentDirections| algorithms for each row of the $E$.
Any form of error metric calculation or subsequent update is performed only after the entire matrix $E$ has been processed using the line-by-line update method.

Since the updated matrix $B$ for the \verb|FrequentDirections| method has constant dimensions throughout the update process, the residual norm error calculation is modified to measure the error between $B$ and $A'$ where $A'$ is a truncated version of $A$ that only holds the first $2l$ singular vectors and values of $A$ and where $2l$ is the number of rows in $B$.

%% DATASETS %%
\subsection{Datasets}

In total, we conducted experiments on five datasets. MED, CRAN, CISI, and Reuters-21578 are term-document matrices from latent semantic indexing applications \cite{LSISite,Cai2005,Cai2007,Cai2008,Cai2009} and ML1M is a movie rating dataset from MovieLens \cite{Harper2015}.
Table \ref{table:datasets} lists the dimensions of the matrices as well as the average number of nonzero (nnz) entries per row and Figure \ref{fig:sv} shows the leading 100 singular values for each matrix.
It should be noted that the matrices used for CISI, CRAN, and MED in \cite{Kalantzis2021} had slightly different dimensions compared to what was listed on \cite{LSISite}.
We received these datasets along with the MATLAB code and chose to use their versions of the data for ease of comparison; as we were interested in the accuracy of singular value reconstruction we determined that somewhat corrupted data merely introduced a different set of singular values to reconstruct.
Furthermore, as the Reuters and ML1M datasets were intact, we used them as controls against the corruption of the other sets.

% Plot of singular value profile plot and table of data matrix stats
\begin{center}
\begin{minipage}[b]{0.4\textwidth}
% \begin{figure}[h]
  \centering
  \includegraphics[width=0.9\textwidth]{figures/sv_100_profile.png}
  \captionof{figure}{Leading 100 singular values for each dataset.}
  \label{fig:sv}
% \end{figure}
\end{minipage}
\hfill
\begin{minipage}[b]{0.58\textwidth}
% \begin{table}[h]
  \centering
  \begin{tabular}{cccc}
    \toprule
    Dataset       & Rows  & Columns & nnz(A)/row \\
    \midrule
    CISI \cite{LSISite}                                     & 5609  &    1460 &      12.17 \\
    CRAN \cite{LSISite}                                     & 4612  &    1398 &      18.06 \\
    MED  \cite{LSISite}                                     & 5831  &    1033 &       8.92 \\
    ML1M \cite{Harper2015}                                  & 6040  &    3952 &     165.60 \\
    Reuters-21578 \cite{Cai2005, Cai2007, Cai2008, Cai2009} & 18933 &    8293 &      20.57 \\
    \bottomrule
    \vspace{5mm}
  \end{tabular}
  \\
  \captionof{table}{Number of rows, columns, and average non-zero elements in each row for datasets.}
  \label{table:datasets}
% \end{table}
\end{minipage}
\end{center}

%% EXPERIMENTS %%
\subsection{Experiments}

We conducted two sets of experiments: one to confirm the results of \cite{Kalantzis2021} in a series of reproducibility studies and another to further measure the performance of the algorithms using two additional metrics as well as observing the effect of the number of batches on the runtime and performance.

\paragraph{Update method comparison} 

As a first step, we sought to reproduce the results in Figures 3 and 4 of \cite{Kalantzis2021}.
To do this, we conducted the sequence updates experiment.
The initial matrix $B\equiv A^{(0)}$ was set equal to the first $\mu$ rows of $A\in\Ctwo{m}{n}$ and the remaining $m-\mu$ rows of $A$ were appended to the initial matrix over a sequence of $\phi$ updates, each with $\tau=\floor{(m-\mu)/\phi}$ rows.
Following the notation of \cite{Kalantzis2021}, the $i$-th update would yield $A^{(i)} = \begin{pmatrix} B\equiv A^{(i-1)} \\ E\equiv A(\mu+(i-1)\tau+1:\mu+i\tau,:) \end{pmatrix}$ with the exception of the last update which is likely to have fewer rows in $E$.
After each update, the rank-$k$ truncated SVD was calculated by one of the three algorithms.

The parameters used in \cite{Kalantzis2021}, and thus in our experiments as well were $\mu=\ceil{m/10}$ rows, $\phi=10$ updates, and rank $k=50$.
The relative errors and residual norms were reported for the $k=50$ leading singular triplets for $\phi=1, 5, 10$.
For Algorithm 2.2, we set the coefficient $\lambda=1.01 \hsigma_1^2$ and $r=10$.

\paragraph{Algorithm 2.2 $r$ parameter study}

Next, we varied the $r$ parameter in Algorithm 2.2 to evaluate its effect on the accuracy as was presented in Table 4 by \cite{Kalantzis2021}.
For this, we set $\mu=\ceil{m/10}$, $\phi=10$, and $k=50$ for all three update methods as with the previous experiment and set $r=10,20,30,40,50$ for Algorithm 2.2.

\paragraph{Runtime comparison} 

We compared the runtimes of the algorithms for the CRAN, CISI, and MED as a function of the rank $k=25,25,50,75,100,125$ and the total number of updates $\phi=2,4,6,8,10$ (Figure 2 left and middle plots in \cite{Kalantzis2021}).

%\paragraph{Additional metrics}
%
%To further compare the performance of the algorithms, we additionally evaluated the accuracy of each method using the projection error \verb|proj_err|$=\norm{A-\hat{A}}_F^2 / \norm{A-A_k}_F^2$ and covariance error \verb|cov_err|=$\norm{A^T A-B^T B}_2 / \norm{A}_F^2$ introduced in \cite{Ghashami2016}.
%These two metrics were used in the \cite{Ghashami2016} to characterize the error between $B$ and $A$.
%
%For the projection error, $\hat{A}$ is defined as the rank $k$ matrix of the projection of $A$ onto $B$. Since $B$ is a subspace approximation of $A$, the projection will only retain the $2l$-dimensional subspace that contains the ``most common'' row space of $A$. Both the projection 
%
\paragraph{Varying number of batches and desired rank}

In addition to the experiments that we replicated based on \cite{Kalantzis2021}, we also varied the number of batches $\phi=2,4,6,8,10$ and the desired rank $k=25,50,75,100,125$ of the truncated SVD and evaluated the performance of each of the update methods to further observe the effects of each of these parameters on the methods' performances.

\section{Data Preprocessing and Augmentation} \label{sec:data}
FixMatch requires a weak augmentation $\alpha(\cdot)$ and a strong augmentation $\mathcal{A}(\cdot)$.
For the weak augmentation, we randomly flip an image with probability $0.5$ as \citep{sohn2020fixmatch} and translate an image up to $12.5\%$ with probability $0.5$ \footnote{Here, \citep{sohn2020fixmatch} didn't indicate what probability they use for the translation.}. For the strong augmentation, FixMatch uses either RandAugment (RA) \citep{cubuk2020randaugment} or CTAugment \citep{kurakin2020remixmatch} for their experiments. However, we use RA for our experiments with the maximum magnitude $10$ (same as the official experiment setup) and $2$ randomly selected operations per image.
% (\RT{ need to justify how 10 is selected})

Due to the limitation of computational resources, we examine the reproducibility of \citep{sohn2020fixmatch} on the dataset CIFAR-10 \citep{krizhevsky2009learning}.
In CIFAR-10, there are $50000$ training data and $10000$ testing data. We take $5000$ training data as the validation dataset. Then we use the remaining training dataset to make labeled and unlabeled datasets and augment both datasets into the same target number as in \citep{sohn2020fixmatch}. After augmentation, we have $2^{13}$ labeled images and $2^{13} \times 7$ unlabeled images for the CIFAR-10 training dataset. 
\section{Experiment}
In the reproducibility experiments, we re-implement FixMatch from scratch using PyTorch and reproduce the essential experiments in the original paper with the similar results.
We use the hyperparameters 
($\lambda_u = 1,\, \eta = 0.03,\, \beta=0.9,\, \tau=0.95,\, \mu = 7,\, B = 64,\, K = 2^{20}$) 
given by \citep{sohn2020fixmatch} and focus on reproducing the performance on CIFAR-10 (Sec. 4.1 of \citep{sohn2020fixmatch}) and barely supervised learning experiments (Sec. 4.4 of \citep{sohn2020fixmatch}). Besides the early introduced hyper-parameters, we use SGD with $\beta = 0.9$ for training the model, and the learning rate is decay with $\eta \cos (\frac{7\pi k}{16 K})$, where $K$ is the total time step and $k$ is the current time step. Each experiment takes around 68 hours on a single V100. And all the error rates is generated from EMA (exponential moving average) of models in the SGD training trajectory.

% Instead of using computational resources for hyperparameter tuning and ablation study (Sec. 5 of \citep{sohn2020fixmatch}), 
% \RT{What we did for reroducibility. A high level description.} 

% the performance of it by replicating similar results shown in the second-last row of Table 2 under section CIFAR-10 in the paper.
Then, we investigate  confirmation errors of "Match"-based SSL methods to see whether there exists such error and asymmetric noise of pseudo labels in FixMatch with the official experiment setup, i.e. unbalanced batches, in  \citep{sohn2020fixmatch}.
Next, we examine the existence of confirmation errors and asymmetric noise for FixMatch again in a more reliable way using re-constructed batches as introduced in Sec. \ref{sec:method}. 
Furthermore, we respectively add the equal-frequency entropy regularization and confidence entropy regularization on the labeled training data in the loss function and compare with the baseline FixMatch without entropy regularization on the BC batches. Finally, we add asymmetric noise to the labeled data in the training dataset and compare the performance of baseline FixMatch and FixMatch with different entropy regularization.

\subsection{Reproducibility} \label{sec:rep}
\paragraph{CIFAR-10.}
We reproduced the experiments on CIFAR-10 with 40, 250, 4000 labeled data and 5000 validation samples as the official implementation of FixMatch\footnote{The official implementation: \url{https://github.com/google-research/fixmatch}. From the reproducibility and readability, the official code is not a valid submission.}.  But due to the limitation of computational resources, we didn't reproduce 5 "folds". Thus, our result based on 1 fold doesn't have the standard deviation. Our model uses the Wide ResNet-28-2 \citep{zagoruyko2016wide} with leaky ReLU activation function. 
Our results are shown in Table \ref{tab:err} which is comparable to the performance in \citep{sohn2020fixmatch}.
% in the sense of lack of computational resources. 
% Note that we didn't run over all the training step and only have $105$ epochs out of total 1024 epochs.  
% \citet{sohn2020fixmatch} claim that FixMatch is the one of few works evaluated on less than 400 labeled training data. 
% Our implementation has reasonably good performance on the dataset with 250 labeled data, however, it has poor performance in the dataset with 40 labeled data. 
% Therefore, our implementation has reasonable results when there are not too few labeled data in training dataset. We further explore the required minimum number of labeled data for our implementation. We found that $20.79\%$ error rate with 100 labeled data in training dataset and $11.5\%$ error rate with 150 labeled data in the training dataset. Therefore, the minimum number of required labeled data in the training dataset can be around 100 ($40 < 100 < 150$) considering the error rate pattern in Table \ref{tab:err}.

\begin{table}[h]
\centering
\small
\caption{ Error rates for CIFAR-10 on test data. FixMatch (RA) uses RandAugment \citep{cubuk2020randaugment}. BC means that the experiment uses balanced-class batches as introduced in Sec. \ref{sec:method}. We use the experiment with BC and RA as a comparison baseline results for the investigation in Sec. \ref{sec:exp_noise}. 
% \lc{It also shows that the performance of our implementation with and without BC batches are barely distinguishable.(we don't have results with BC batches)}
} \label{tab:err}
\begin{tabular}{cccccc}
\hline
              & \multicolumn{3}{c}{CIFAR-10} \\ \hline
Method        & 40 labels     & 250 labels     & 4000 labels     \\ \hline
Official FixMatch (RA) & $13.81 \pm 3.37$      & $5.07\pm 0.65$        &     $4.26\pm 0.05$     \\ 
Ours (RA) & $10.04$     & $5.29$        &     $4.36$     \\ 
% Ours (RA+BC) & $39.85 \sim 40.10$   & $10.84 \sim 11.07$        &     $6.33 \sim 6.57$    \\ 
\hline
\end{tabular}
\end{table}

\paragraph{Barely supervised learning.}
We also reproduce the one example per class experiment.
\citep{sohn2020fixmatch} hypothesize that the repressiveness of the chosen labeled data influences the results significantly. Since there are only one/few samples per class, this hypothesis is reasonable intuitively. Then, \citet{sohn2020fixmatch} categorized the training dataset into eight levels of “prototypicality”, i.e., representative of the underlying class and then ordered the training samples by their “prototypicality”. With the same hyperparameters, the model is trained with $10$ provided most representative labeled data under Random Augment. The accuracy is $84.41\%$ compared with the official implementation: a median of $78\%$ accuracy and a maximum of $84\%$ accuracy. 
% Since the minimum number 100 of labeled data for our implementation is larger than the one 40 of the official implementation, we hypothesize that for the barely supervised learning experiment we also need more labeled data than 10. Unfortunately, we didn't have more time to spend on choosing more representative labels for the experiment and more computational resources for this specific experiment.

\subsection{Ablation studies}
The ablation studies are based on FixMatch with 250 labels using CTAugment. 
\paragraph{Study for Confidence threshold.}
We performance the ablation studies for confidence threshold. Due to the limited computation resource, we hypothesize that experiments with lower confidence threshold will achieve worse performance and explore more values around the optimal value of confidence threshold, 0.95 chosen by the authors. Thus our examined threshold value is between 0.7 to 0.98. As shown in Figure\ref{fig:abastudy} (c), the error rate is between $6.54\%$ and $6.19\%$ and the highest performance is under the threshold 0.98.  

\begin{figure}[h]
\centering
\includegraphics[width=14cm]{../openreview/fig/FixMatch.png}
\caption{Plots of various ablation studies on FixMatch compared with those reported in the paper. (a) Varying the confidence threshold for pseudo-labels in the original paper. (b) Varying the ratio of unlabled data ($\mu$) in the original paper. (c)Varying the confidence threshold for pseudo-labels based on our implementation. (d) Varying the ratio of unlabled data ($\mu$) based on our implementation.
}\label{fig:abastudy}
\end{figure}



\paragraph{Ratio of unlabeled data.}

We perform FixMatch under different ratios of unlabeled data. Figure\ref{fig:abastudy} (d) shows the error rate which is decreasing when the ratio of unlabeled data is higher. A significant increase of the accuracy happens using a large number of unlabeled data. The results show the consistency with the finding in the original paper.

% \begin{figure}[h]
% \centering
% \includegraphics[width=14cm]{fig/ablation.png}
% \caption{
% }\label{fig:abastudy}
% \end{figure}


\subsection{Investigation on confirmation errors and asymmetric noisy (pseudo) labels} \label{sec:exp_noise}
In this section, we show the investigation on confirmation errors and asymmetric noise in labels and pseudo labels and whether the entropy regularization in loss functions \eqref{eq:eqr} and \eqref{eq:cer} can deal with them and improve the performance of FixMatch. The training dataset contains 150 labeled data before augmentation and each BC batch in the training phase contains images with uniformly distributed classes.

\paragraph{Existence of asymmetric noise and confirmation errors in pseudo labels.} We examine the existence of asymmetric noise in pseudo labels by checking the confusion matrix of the prediction of unlabeled data in different batches. Top figures of Figure \ref{fig:cfmx} show the confusion matrices in the experiments without using BC batches. We find that asymmetric noise appears in a random manner, which is as our expectation as analyzed in Sec. \ref{sec:method}. The stochastic behavior is inherited from the randomness of batch composition. Next, we evaluate the asymmetric noise with BC batches, which is a more reliable way as mentioned in Sec. \ref{sec:method}. We found that there exists consistent asymmetric noise, which leads to the confirmation errors, i.e., the model always tends to wrongly predict certain images into certain classes as shown in bottom figures of Figure \ref{fig:cfmx}.
Moreover, the accuracy of our implementation is $93.6\%$ without BC batches and $93.8\%$ with BC batches, which shows that using BC batches has rarely influence on the model performance compared with the one without BC batches.

\begin{figure}[h]
\centering
\includegraphics[width=14cm]{../openreview/fig/uBC.png}\\
\includegraphics[width=14cm]{../openreview/fig/BC.png}
\caption{Confusion matrices of the confident prediction on unlabeled data with different batch structures.
Confusion matrices are plotted every 100 training steps in the 1st epoch (1024 steps).
The \textbf{top} matrices are from the experiments without BC, and the \textbf{bottom} matrices are the ones with BC. The red areas represent the asymmetric noise in the pseudo labels. The bottom matrices have a stable and smooth transition while the top matrices have a fluctuating transition in the red areas. The yellow color represents larger value and the darker green color represents smaller values.
}\label{fig:cfmx}
\end{figure}

\paragraph{Equal-frequency and confidence entropy regularization on the labeled data.}
Due to limitation of the computational resources, we didn't explicitly run grid search for the hyperparameters in the Equal-Frequency (EF) loss function  \eqref{eq:eqr} and Confidence-Entropy (CE) loss function \eqref{eq:cer}. Instead, we found that for the baseline method the training loss is around 0.2. We then compute the equal-frequency entropy loss for the ideal scenario, equal frequency for all classes, which is $ 0.1 \times \ln 0.1 \approx 2 $. We decide to try the hyperparameter $\lambda_{ce}$, $\lambda_{ef}\leq 0.1$ to avoid making the entropy regularization loss dominate the loss value. Then, we do a hyper-parameter search for the loss function \eqref{eq:eqr} and \eqref{eq:cer}. For all experiments in this experiment, we used cosine function decay for the parameters $\lambda_{ce}$ and $\lambda_{eq}$, which starts with value $1$ and ends with value $0$ in the training phase. We find that using the loss function \eqref{eq:cer} can achieve a better accuracy performance $94.01\%$. Moreover, as an advantage, using the confidence entropy regularization can reduce the asymmetric noise as shown in the bottom confusion matrices of Figure \ref{fig:cfmx_eta}. 
As for the equal-frequency entropy regularization, it has a better accuracy, $93.85\%$. Moreover, the equal-frequency entropy regularization can penalize the asymmetric noise, which may transform it into symmetric noise as shown in the middle confusion matrices of Figure \ref{fig:cfmx_eta}. Note that there are plenty of ways to deal with symmetric noise, which is much easier to handle.

\begin{table}[!h]
\centering
\caption{Error rates on testing data using the loss function \eqref{eq:eqr} and \eqref{eq:cer}. The experiments use 150 labeled data and CTA for training. The first column is the results without BC batch and the second column is the baseline result without using EF or CE regularization.}\label{tab:hp_e}
\begin{tabular}{ccccc}
\hline
Entropy regularization & noBC+Null & BC+Null   & BC+CE    & BC+EF \\ \hline
$\lambda_{ce}/\lambda_{ef}$ & 0  &0  &    0.1 & 0.1 \\ \hline
Error rate    & 6.4 &6.2&    \textbf{5.99}    & 6.15 \\ \hline
\end{tabular}
\end{table}

\begin{figure}[h]
\centering
\includegraphics[width=14cm]{../openreview/fig/eta0.png}\\
\includegraphics[width=14cm]{../openreview/fig/2eta005.png}\\
\includegraphics[width=14cm]{../openreview/fig/eta005.png}
\caption{Confusion matrices of the confident prediction on unlabeled data with BC batches using loss functions \eqref{eq:bsl} without entropy regularization at \textbf{top}, 
\eqref{eq:eqr} with equal-frequencey entropy regularization in the \textbf{middle},
and \eqref{eq:cer} with confidence entropy regularization at \textbf{bottom}. Confusion matrices are plotted every 100 training steps in the 1st epoch (1024 steps). The red areas represent the asymmetric/symmetric noise in the pseudo labels. The yellow color represents larger value and the darker green color represents smaller values.
}\label{fig:cfmx_eta}
\end{figure}

\paragraph{Equal-frequency and confidence entropy regularization on the labeled data containing asymmetric noise.}
In this experiment, we use RA data augmentation and manually add asymmetric noise to the labeled data in the training dataset to compare how FixMatch with different loss functions performs in the presence of asymmetric noise in the labeled data.
We respectively select $3$ images from class $0$ and class $1$ in the validation dataset. Then, for the labeled data in the training dataset, we keep the labels unchanged and replace $3$ images in class $2$ with the $3$ images in class $0$. Similarly we replace $3$ images in class $3$ with the $3$ images in class $1$. In this way, the only difference with the previous experiments in this section is that our final validation dataset has $4994$ images and the labeled data in the training dataset contain asymmetric noise. 
Table \ref{tab:ans} shows error rates on 6 runs with different random seeds.
In the presence of asymmetric noise in labeled training data, all proposed methods perform better than the baseline method, in which FixMatch with BC batches decreases the average error rate from 8.6 to 7.37, and the combination of confidence-entropy regularization and BC batches further lowers the error rate to 6.98.
\begin{table}[!h]
\centering
\caption{Error rates of FixMatch methods in the presence of asymmetric noise in labeled training data augmented by RA: 
The baseline method ($\lambda=0$);
The method ($\lambda=0$) with BC batches;
the method with confidence-entropy regularization ($\lambda_{ce}=0.1$) and BC batches;
the method with equal-frequency regularization ($\lambda_{ef}=0.1$) and BC batches.
% The second column is the accuracy of the baseline FixMatch, the third column is the accuracy of FixMactch using equal-frequency regularization, and the last one is the accuracy of FixMatch using confidence entropy regularization after 60 epochs.
}\label{tab:ans}
\begin{tabular}{cccc|c}
\hline
          & $\lambda=0$(noBC) & $\lambda=0$(BC)   & $\lambda_{ef}=0.1$(BC)   & $\lambda_{ce}=0.1$(BC)\\ \hline
Error rates on test data   & $8.6 \pm 2.81$ & $7.37 \pm 2.05$ & $7.95 \pm 2.2$  & \textbf{6.98 $\pm$ 1.83} \\ \hline
\end{tabular}
\end{table}

% 4) we only use the noise-robust loss function for pseudo labels of unlabeled data without balanced-class batches. Only for the pseudo labels because we believe the data in CIFAR-10 do not have noisy labels. 
% \RT{We also need to the hyper-parameter tuning for the loss function in Eqn. \eqref{eq:nrloss}.}


\section{Challenges}
% Challenges: Challenges you faced when reimplmenting the paper and conducting the experiments. Were all details in the paper? Or did you have to look in the authors code or even contact them to find about some details? Was parts of the code quite hard to get them to work as intended? Did you have optimize and tune several hyperparameters? Which ones? Did the framework you used  make the implementation difficult in some ways?
It is not clear how many steps are there in each epoch. First the paper only states the total steps $K=2^{20}$ and the composition of one batch ($B$ labeled samples and $\mu B$ unlabeled samples). And the official code indicates there are $2^{16}$ labeled images observed by the model per epoch and a total of $2^{26}$ images observed which suggests that there are $2^{12}$ updates per epoch and $2^{19}$ updates in total. And this is not consistent with the total update steps $K$ stated in the paper. When performing weak augmentations to the input data, the probability for randomly translating images is not specified. And it also remains unclear the `5 different folds' mentioned in the paper, we are guessing it is a kind of cross validation while there is not too much evidence supporting this neither in the paper nor in the official code.

The paper doesn't contain sufficient details to reproduce all the experiments. Thus, it is necessary to look for details about reproducing the experiments in the official code. We have not optimized or tuned the hyperparameters, and all the hyperparameters are the same as those mentioned in the paper. 
Compared to the average error rates in the original paper, the reproduced results have a reasonable good performance on a larger number of labeled data (4000/250 labels) and better but also reasonable performance on fewer labeled data (40/10 labels) since the variance of error rates over 5 different folds for CIFAR-10 with 40 labels is $3.35\%$. 
% Some different implementations are considered. One implementation difference could be that we did not add weight decay parameters to the SGD optimizer. 
% Another implementation difference is that we used post-shifted Batch Normalization (PSBN) proposed in \citep{zhou2020does} instead of standard Batch Normalization (BN) in the wideResNet, aiming to prevent undesirable collapsed filters that are common in Deep Neural Networks with BN. Due to the time limitation, we didn't compare the performance of the model under PSBN and standard BN. 
% Moreover, in our experiment, we only consider RAugment that the magnitude is randomly selected in the predefined range, while the magnitude is learned during training in CTAugment.
Moreover, to compare with the results of ablation studies in the original paper, we also implement CTAugment, which supports a learnable magnitude. While we failed to confirm the result that CTAugment behaves better than RandAugment on CIFAR-10. We hypothetically guess this is because it could affect the consistency regularization because of different levels of distortions controlled by magnitude. 
\section{Conclusion}
In this work, we study and reimplement FixMatch from scratch. We reproduced essential experiments, included the model performance on CIFAR 10, barely supervised learning, and ablation studies.
Experimental results show that our implementation achieves similar performance as the original FixMatch results, which supports that FixMatch outperforms semi-superivesed learning benchmarks and that the author's choices with respect to those ablations were experimentally sound. 
We also confirmed the existence of confirmation errors in pseudo labels by checking the prediction confusion matrix of unlabeled data in different training stages.
% After that, 
% We analyze the training procedure and figure out the potential severe confirmation errors. 
% As a harmful and tricky source of the confirmation errors, asymmetric noise in the (pseudo) labels is thoroughly investigated. 
We adapted the training batch structure to be composed of equal number of images in each class, which enable us to stably and reliably check the the asymmetric noise in the training phase. 
% It shows that there exists asymmetric noise in pseudo labels of unlabeled data causing confirmation error. 
Based on the reconstructed batch structure, we used the equal-frequency and confidence entropy regularization in the loss function, and theoretically show their relation. 
The experiments indicate that these entropy regularization can reduce the asymmetric noise in pseudo labels and improves the performance of FixMatch in the presence of training labels with asymmetric noise.

\section{Ethical consideration}
The bias in the collected dataset is a serious problem when applying machine learning methods to the real-world scenarios. For example, applying machine learning methods to making automated decision-making systems for criminal prediction, university admission or recruitment. In these cases, we may very likely collect a dataset containing certain bias due to the historical reason or selection bias in the data collection process. If a model cannot deal with such bias in the dataset, it may inherit in the model by focusing on the unrelated or wrong relations in the dataset. Consequently, the model can make biased decision which can disadvantage a certain group of people and may even diminish this group in the society. 

Unfortunately, FixMatch cannot only be influenced by the noise in the label of a training dataset, but also it can make confirmation errors causing a biased model even when the dataset itself is unbiased. To deal with such issue, this work focuses on the asymmetric noise in the data labels and pseudo labels, which can lead to severe confirmation error and the biased model. And then, we applied different methods to reduce such noise in pseudo labels and reduce its impact on the model.

% \input{self}
%\bibliography{ref.bib}
%\bibliographystyle{abbrvnat}
