\section*{\centering Reproducibility Summary}

\subsection*{Scope of Reproducibility}
The main objective of the paper is to "learn the Hamiltonian dynamics of simple physical systems from high-dimensional observations without restrictive domain assumptions".
To do so, the authors train a generative model that reconstructs an inputted sequence of images of the evolution of some physical system.
For instance, they learn the dynamics of a pendulum, a body-spring system, and 2,3-bodies.
In addition to these environments, we further expand the testing on two new environments and we explore architecture tweaks looking for performance gains.

\subsection*{Methodology}
We implement the project with Python using Pytorch \cite{pytorch} as a deep learning library.
Previous to ours, there was no public implementation of this work.
Thus, we had to write the code of the simulated environments, the deep models, and the training process.
The code can be found in this repository: \href{https://github.com/CampusAI/Hamiltonian-Generative-Networks}{https://github.com/CampusAI/Hamiltonian-Generative-Networks}
A single training takes around 4 hours and 1910MB of GPU memory (NVIDIA GeForce RTX2080Ti).


\subsection*{Results}
We found the model's input-output data slightly unclear in the original paper.
First, it seems that the model reconstructs the same sequence that has been inputted.
Nevertheless, further discussion with the authors seems to indicate that they input the first few frames to the network and reconstructed the rest of the rollout.
We test both approaches and analyze the results.
We generally obtain comparable results to those of the original authors when just reconstructing the input sequence ($30\%$ average absolute relative error w.r.t. to their reported values) and worse results when trying to reconstruct unseen frames ($107\%$ error).
In this report, we include our intuition on possible reasons that would explain these observations.


\subsection*{What was easy}
The architecture of the model and training procedure was easy to understand from the paper.
Besides, creating simulation environments similar to those of the original authors was also straightforward. 

\subsection*{What was difficult}
While the overall model architecture and data generation were easy to understand, we encountered the optimization to be especially tricky to perform.
In particular, finding a good balance between the reconstruction loss and KL divergence loss was challenging.
We implemented GECO \cite{geco} to dynamically adapt the Lagrange multiplier but it proved to be surprisingly brittle to its hyper-parameters, resulting in very unstable behavior.
We were unable to identify the cause of the problem and ended up training with simpler techniques such as using a fixed Lagrange multiplier as presented in \cite{beta-vae}.

\subsection*{Communication with original authors}
We exchanged around 6 emails with doubts and answers with the original authors.


\newpage

\section{Introduction}

Most prominent vision datasets are afflicted by \emph{contextual bias}. For example, ``microwave" typically is found in kitchens, which also contain objects like ``refrigerator" and ``oven."  Such co-occurrence patterns may inadvertently induce contextual bias in datasets, which could consequently seep into models trained on them. When models overly rely on context, they may not generalize to settings where typical co-occurrence patterns are absent. The original paper by Singh et al.~\cite{Singh_2020_CVPR} proposes two methods for mitigating such contextual biases and improving the robustness of the learnt feature representations. The paper demonstrates their methods on multi-label object and attribute classification tasks, using the COCO-Stuff~\cite{caesar2018cvpr}, DeepFashion~\cite{liuLQWTcvpr16DeepFashion}, Animals with Attributes (AwA)~\cite{AwA}, and UnRel~\cite{Peyre17} datasets. Our exploration centers on four main directions:\\
\\
First, we trained the baseline classifier presented in the paper (Section~\ref{sec:baselineimplementation} for implementation and training details; Sections~\ref{sec:evaluation}-\ref{sec:baselineresults} for results). Due to likely implementation discrepancies, our results differed from the original paper by 0.6--3.1\% mAP on COCO-Stuff, by 0.7--1.4\% top-3 recall on DeepFashion, and by 0.1--3.2\% mAP on AwA (Table~\ref{tab:mainresults}). We ran a hyperparameter search (Appendix~\ref{sec:hyperparametersearch}), which yielded a significant (1.4--3.6\%) improvement on DeepFashion.\\
\\
Next, we identified the \emph{biased categories} in each dataset, i.e., visual categories that suffer from contextual bias. We followed the proposed method of using the baseline classifier to identify these categories, and discovered that the classifier implementation has a non-trivial effect. For COCO-Stuff, 18 of the top-20 categories we identified matched the original paper's top-20 categories (10 on DeepFashion, 18 on AwA; Section~\ref{sec:biasedcategories}). Nevertheless, the categories we identified appear reasonable  (e.g., ``fork" co-occurs with ``dining table"; Appendix~\ref{sec:biasedcategoriesapp}). As training and evaluation of most methods depend on the biased categories, we used the paper's biased categories for subsequent experiments.\\
\\
Third, we checked the main claim of the paper, that the proposed \emph{CAM-based} and \emph{feature-split} methods help improve recognition of biased categories in the absence of their context (Section~\ref{sec:stage2}). On COCO-Stuff, DeepFashion, and UnRel, we were able to reproduce the improvements gained from the proposed \textit{feature-split} method towards reducing contextual bias, whereas on AwA, we saw a drop in performance. The proposed \textit{CAM-based} method, which was only applied to COCO-Stuff, also helped reduce contextual bias, though not as significantly as the \textit{feature-split} method. For the  method, we reproduced the original paper's results to within 0.5\% mAP (Section~\ref{sec:mainresults}). We also successfully reproduced the paper's weight similarity analysis, as well as the qualitative analyses with class activation maps (CAMs)~\cite{zhou2015cnnlocalization}.\\
\\
Lastly, we ran additional experiments and ablation studies (Section~\ref{sec:addanalyses}). These revealed that the regularization term in the \emph{CAM-based} method and the weighted loss in the \emph{feature-split} method are central to the methods' performance. We also observed that varying the feature subspace size influences the \emph{feature-split} method's accuracy.


\section{Scope of reproducibility}
\label{sec:claims}

\jdcomment{Explain the claims from the original paper you picked for the reproduction study and briefly motivate your choice. Try to summarize each claim  in 1-2 sentences, e.g. "The introduced activation function X outperforms a similar activation function Y on tasks Z,V,W". Make the scope as specific as possible. It should be something that can be supported or rejected by your data. For example, this scope is too broad and lacks precise outcome: "Contextual embedding models have shown strong performance on a number of tasks across NLP. We will run experiments evaluating two types of contextual embedding models on datasets X, Y, and Z." This scope is better because it's more specific and has an outcome that can be either supported or rejected based on your results: "Finetuning pretrained BERT on SST-2 will have higher accuracy than an LSTM trained with GloVe embeddings."}

In order to verify the central claims presented in the paper we focus on the following target questions:

\begin{itemize}
    \item Does \textit{RigL} outperform existing sparse-to-sparse training techniques---such as SET (\citet{Mocanu2018SET}) and SNFS (\citet{dettmers2020sparse})---and match the accuracy of dense-to-sparse training methods such as iterative pruning (\citet{to_prune_or_not})?
    
    \item \textit{RigL} requires two additional hyperparameters to tune. We investigate the sensitivity of final performance to these hyperparameters across a variety of target sparsities (Section \ref{hyperparameter-tuning}).
    
    \item How does the choice of sparsity initialization affect the final performance for a fixed parameter count and a fixed training budget (Section \ref{effect-sparsity-distribution})?
    
    \item Does redistributing layer-wise sparsity during connection updates (\citet{dettmers2020sparse}) improve \textit{RigL}'s performance? Can the final layer-wise distribution serve as a good sparsity initialization scheme (Section \ref{effect-redistribution})? 

\end{itemize}

\jdcomment{Each experiment in Section~\ref{sec:results} will support (at least) one of these claims, so a reader of your report should be able to understand the \emph{arguments} (claims) and, separately, the \emph{evidence} that supports them.}
%\jdcomment{To organizers: I asked my students to connect the main claims and the experiments that supported them. For example, in this list above they could have ``Claim 1, which is supported by Experiment 1 in Figure 1.'' The benefit was that this caused the students to think about what their experiments were showing (as opposed to blindly rerunning each experiment and not considering how it fit into the overall story), but honestly it seemed hard for the students to understand what I was asking for.}

\section{METHODOLOGY} \label{sec:methodology}
% Explain your approach - did you use the author's code, or did you aim to re-implement the approach from the description in the paper? Summarize the resources (code, documentation, GPUs) that you used.
The original implementation of the CGN is publicly available \cite{cgn-github}, but most of the experiments conducted in the original paper to support their claims are not. Consequently, we use the authors' code for the implementation of the CGN, and re-implement the experiments and relevant evaluation metrics based on the descriptions provided in the paper. Furthermore, we both improve and extend upon the work of \citeauthor{Sauer2021ICLR} by providing additional experiments and results. Because a description of the GAN used in the original paper was not provided, we use a DCGAN \cite{dcgan-tutorial}.
% \footnote{\url{https://pytorch.org/tutorials/beginner/dcgan_faces_tutorial.html}}. 

% The main model and training code was released publicly\footnote{\href{https://github.com/autonomousvision/counterfactual\_generative\_networks}{https://github.com/autonomousvision/counterfactual\_generative\_networks}}. For certain baselines, the code was not part of the released repository. We used code openly available from the original papers and adapted it to a setting as required by this paper \cite{Sauer2021ICLR}. For specific cases, (e.g. training a vanilla GAN on MNIST variants), we implemented these using specific resources (e.g. \href{https://pytorch.org/tutorials/beginner/dcgan_faces_tutorial.html}{PyTorch's DCGAN tutorial}). Further, we implemented all experiments using their provided code as an API and the details provided in the paper. As compute resources, we primarily used a single-node GPU equipped with a 1080Ti for all our experiments with some exceptions (e.g. classifiers on MNISTs) trained on local CPU machines.

% The two key tasks central to the paper are: (i) generating counterfactual examples, (ii) training classifiers to be invariant to spurious correlations. For (i), \citewithauthor{Sauer2021ICLR} propose a deep generative model disentangled into \textit{independent mechanisms} (e.g. shape, texture, background). This enables us to change one factor-of-variation (FoV) at-a-time to obtain counterfactual examples. Once such examples are generated, we can train classifiers invariant to spurious correlations. In the remainder of section, we provide details regarding models (\cref{subsec:model}), datasets (\cref{subsec:datasets}) and the experimental setup (\cref{subsec:expt}).

\subsection{Datasets} \label{subsec:datasets}
% For each dataset include 1) relevant statistics such as the number of examples and label distributions, 2) details of train / dev / test splits, 3) an explanation of any preprocessing done, and 4) a link to download the data (if available).
The experiments conducted in the original paper involve two tasks, namely generating counterfactual examples and training a classifier to be invariant to spurious correlations. We follow the paper and reproduce their evaluations on multiple datasets for each task. For both tasks, we present the relevant datasets and their main purpose in \cref{tab:datasets}. Due to resource constraints, running all experiments on full ImageNet (IN-1k) is infeasible. As a compromise, we use ImageNet-mini (IN-Mini) \cite{in-mini}, a small-scale variant of ImageNet. Although this dataset contains fewer samples, we found it to be sufficient to reproduce the main findings of the original paper and verify their claims. Moreover, this dataset includes the same classes as IN-1k and hence does not induce any decrease in difficulty of the classification task.

%%%%%%%% Dataset table: V1 %%%%%%%%%%%%%
\begin{table}[H]
\centering
\setlength{\aboverulesep}{1.2pt}
\setlength{\belowrulesep}{1.2pt}
\scriptsize
\caption{\textbf{Datasets overview.} The datasets used for empirical evaluations across two tasks.}
\label{tab:datasets}
\resizebox{\textwidth}{!}{%
\begin{tabular}{llrrrrll}
\toprule
\textbf{Task} & \textbf{Datasets} & \multicolumn{3}{c}{\textbf{Number of samples}} & \textbf{Classes} & \textbf{Description} & \textbf{URL} \\ \cmidrule{3-5}
 &  & Train & Test & Total &  &  &  \\
 \midrule
\multirow{4}{*}{\begin{tabular}[c]{@{}l@{}}Generating\\ counterfactual\\ samples\end{tabular}} & C-MNIST \cite{IRM} & 50k & 10k & 60k & 10 & Foreground colour as a spurious correlation & \href{https://github.com/autonomousvision/counterfactual_generative_networks/blob/main/scripts/download_data.sh}{Link} \\
\arrayrulecolor{lightgray}\cmidrule{2-8}
 & DC-MNIST & 50k & 10k & 60k & 10 & Fore/background colour as spurious correlations & NA\footnotemark[1]
 \\
\arrayrulecolor{lightgray}\cmidrule{2-8}
 & W-MNIST & 50k & 10k & 60k & 10 & In-the-wild background with texture colour & NA\footnotemark[1] \\
\arrayrulecolor{lightgray}\cmidrule{2-8}
 & IN-1k \cite{imagenet-1m} & 1M & 100k & 1.2M & 1000 & Large-scale evaluation & \href{https://www.kaggle.com/c/imagenet-object-localization-challenge/data}{Link} \\
\arrayrulecolor{lightgray}\cmidrule{2-8}
 & IN-mini \cite{in-mini} & ~35k & ~4k & ~39k & 1000 & Small-scale evaluation & \href{https://www.kaggle.com/ifigotin/imagenetmini-1000}{Link} \\
\arrayrulecolor{black}\midrule
\multirow{3}{*}{\begin{tabular}[c]{@{}l@{}}Training\\ invariant\\ classifiers\end{tabular}} & MNISTs & 50k & 10k & 60k & 10 & Test different granularities of invariance &  \href{https://github.com/autonomousvision/counterfactual_generative_networks/blob/main/scripts/download_data.sh}{Link} \\
\arrayrulecolor{lightgray}\cmidrule{2-8}
 & Cue-conflict \cite{cue_conflict} & NA & 1280 & 1280 & 16 & Tests shape-texture disentanglement & \href{https://github.com/rgeirhos/texture-vs-shape}{Link}  \\
\arrayrulecolor{lightgray}\cmidrule{2-8}
 & IN-9 variants \cite{imagenet9} & $\sim$45k & $\sim$4k & $\sim$50k & 9 & Tests background-invariance &  \href{https://github.com/MadryLab/backgrounds_challenge}{Link} \\
\arrayrulecolor{black}\bottomrule
\end{tabular}%
}
\end{table}
\footnotetext[1]{This variant of MNIST is generated by the authors themselves and can be generated using their repository.}
%%%%%%%%%%%%% %%%%%%%%%%%%% %%%%%%%%%%%%%

\subsection{Hyperparameters}
% Describe how the hyperparameter values were set. If there was a hyperparameter search done, be sure to include the range of hyperparameters searched over, the method used to search (e.g. manual search, random search, Bayesian optimization, etc.), and the best hyperparameters found. Include the number of total experiments (e.g. hyperparameter trials). You can also include all results from that search (not just the best-found results).
In order to match the original experiments as closely as possible, we used the same hyperparameters as the authors of the original paper whenever they were specified in the article. If the required hyperparameters for the experiments were not mentioned in the original paper, we relied on the default parameters given in the configuration files of the original implementation. In this case, we assume that these default parameters correspond to the parameters used for the described experiments.

\subsection{Experimental setup and evaluation metrics}
\label{subsec:expt}
% Include a description of how the experiments were set up that's clear enough a reader could replicate the setup. 
% Include a description of the specific measure used to evaluate the experiments (e.g. accuracy, precision@K, BLEU score, etc.). 
% Provide a link to your code.
Our experimental setup is largely based on the description provided by \citewithauthor{Sauer2021ICLR}. To that end, we will address claim HQC by performing a qualitative analysis on both MNIST and ImageNet. 
To verify claim IBR, we perform a loss ablation study in which we disable one loss at a time. Lastly, to address the main claim of the paper, namely ODR, we conduct a number of experiments on both MNIST and ImageNet to evaluate both out-of-distribution performance and spurious signal invariance of the invariant classifiers. 

To provide further evidence to support claim ODR, we conduct additional experiments to visually explain the decisions made by the invariant classifiers based on gradient-based localization. For this purpose, we use a PyTorch implementation of GradCAM \cite{torchcam, gradcam}, a class activation map method that weighs the 2D activations by the average gradient \cite{gradcam}. This method allows us to visualize the salient features on which the invariant classifiers base their predictions.


\subsection{Computational requirements}
% Include a description of the hardware used, such as the GPU or CPU the experiments were run on. 
% For each model, include a measure of the average runtime (e.g. average time to predict labels for a given validation set with a particular batch size).
% For each experiment, include the total computational requirements (e.g. the total GPU hours spent).
% (Note: you'll likely have to record this as you run your experiments, so it's better to think about it ahead of time). Generally, consider the perspective of a reader who wants to use the approach described in the paper --- list what they would find useful.

We perform all experiments on a cluster whose nodes are equipped with Nvidia GeForce GTX 1080 Ti GPUs. 
% % TODO: Running time of experiments (I am not sure about this).
Due to constraints in resources, we run most experiments once. As such, our experiments are indicative and not conclusive. Our reproducibility study comes at a total computational cost of 112 GPU hours (see \cref{sec:cost-taxonomy} for more details).

% We perform all experiments on GPU nodes of Lisa Cluster, which is a system maintained by SURFsara \cite{Lisa}. These nodes are equipped with Nvidia GeForce GTX 1080 Ti GPUs and Intel Xeon Bronze 3104 CPUs. 
% % TODO: Running time of experiments (I am not sure about this).
% Due to constraints in resources, we run most experiments once. As such, our experiments are merely indicative and not conclusive. Our reproducibility study comes at a total computational cost of 112 GPU hours (see \cref{sec:cost-taxonomy} for our calculations).

% Start with a high-level overview of your results. Does your work support the claims you listed in section 2.1? Keep this section as factual and precise as possible, reserve your judgement and discussion points for the next "Discussion" section. 

% Go into each individual result you have, say how it relates to one of the claims and explain what your result is. Logically group related results into sections. Clearly state if you have gone beyond the original paper to run additional experiments and how they relate to the original claims. 

% \subsection{Result 1}

% \subsection{Result 2}

% \subsection{Additional results not present in the original paper}

\section{Results} \label{sec:reproduction}

We first test whether the HGN \cite{hgn} can learn the dynamics of the four presented physical systems by measuring the average mean squared error (MSE) of the pixel reconstructions of each predicted frame.
Furthermore, we test the original HGN architecture along with different modifications: a version trained with Euler integration rather than Leapfrog integration (HGN Euler),
%a version trained with no transformer network (HGN no $f_\psi$),
and a version that does not include sampling from the posterior $q_\phi(\bm{z}|\bm{x}_0 ... \bm{x}_T)$ (HGN determ). Since we could not find suitable GECO\cite{geco} hyperparameters, we use a fixed Lagrange multiplier\cite{beta-vae} in all the experiments.
%To test HNN, we use the implementation provided by \cite{hnn} known as pixelHNN.
%Similarly to \cite{hgn}, we test HNN with a modification of the architecture to closely match the HGN (HNN Conv).

\begin{table}[]
    \centering
    \resizebox{\textwidth}{!}{%
    \begin{tabular}{c c c c c c c c c}
     \Xhline{3\arrayrulewidth}
     \textsc{Model} & \multicolumn{2}{c}{\textsc{Mass-spring}} & \multicolumn{2}{c}{\textsc{Pendulum}} & \multicolumn{2}{c}{\textsc{Two-body}} & \multicolumn{2}{c}{\textsc{Three-body}} \\
     & \textsc{Train} & \textsc{Test} & \textsc{Train} & \textsc{Test}& \textsc{Train} & \textsc{Test}& \textsc{Train} & \textsc{Test}\\
    \Xhline{3\arrayrulewidth}
     
    Orig. \textsc{HGN (Euler)} \cite{hgn} & $3.67 \pm 1.09$ & $6.2 \pm 2.69$ & $5.43 \pm 2.53$ & $10.93 \pm 4.32$ & $6.62 \pm 3.93$ & $15.06 \pm 7.01$ & $7.51 \pm 3.49$ & $9.4 \pm 3.92$ \\
    Orig. \textsc{HGN (Determ)} \cite{hgn} & $0.23 \pm 0.23$ & $3.07 \pm 1.06$ & $0.79 \pm 1.24$ & $10.68 \pm 3.19$ & $2.34 \pm 2.3$ & $14.47 \pm 5.24$ & $4.1 \pm 2.05$ & $5.17 \pm 1.96$ \\
    %Orig. \textsc{HGN (No $f_\psi$)} \cite{hgn} & $4.95 \pm 1.71$ & $7.04 \pm 2.55$ & $6.83 \pm 3.29$ & $13.98 \pm 4.94$ & $6.35 \pm 3.86$ & $16.49 \pm 6.6$ & $8.37 \pm 3.13$ & $10.41 \pm 3.72$ \\
    Orig. \textsc{HGN (Leapfrog)} \cite{hgn} & $3.84 \pm 1.07$ & $6.23 \pm 2.03$ & $4.9 \pm 1.86$ & $11.72 \pm 4.14$ & $6.36 \pm 3.29$ & $16.47 \pm 7.15$ & $7.88 \pm 3.55$ & $9.8 \pm 3.72$ \\
     \Xhline{3\arrayrulewidth}
     \textsc{HGN (Euler)} ours & $9.05 \pm 0.02$ & $9.06 \pm 0.05$ & $17.79 \pm 0.06$ & $17.86 \pm 0.13$ & $3.84 \pm 0.01$ & $3.85 \pm 0.02$ & $1.99 \pm 0.01$ & $1.99 \pm 0.01$ \\
     \textsc{HGN (Determ)} ours & $7.10 \pm 0.01$ & $7.10 \pm 0.03$ & $14.11 \pm 0.05$ & $14.14 \pm 0.12$ & $3.92 \pm 0.02$ & $3.93 \pm 0.02$ & $4.14 \pm 0.01$ & $4.13 \pm 0.02$ \\

     \textsc{HGN (Leapfrog)} ours & $7.11 \pm 0.01$ & $7.12 \pm 0.03$ & $14.89 \pm 0.05$ & $14.97 \pm 0.1$ & $3.36 \pm 0.01$ & $3.36 \pm 0.02$ & $8.81 \pm 0.01$ & $8.81 \pm 0.01$ \\
     \Xhline{3\arrayrulewidth}
     \textsc{HGN (Euler)} ours \textit{5-frame inference} & $42.09\pm 0.14$ & $41.98\pm 0.32$ & $47.06\pm 0.17$ & $47.03\pm 0.39$ & $6.46\pm 0.03$ & $6.52 \pm 0.06$ & $8.18 \pm 0.01$ & $8.17 \pm 0.01$ \\
     \textsc{HGN (Determ)} ours \textit{5-frame inference}& $13.00 \pm 0.05$ & $13.04 \pm 0.11$ & $45.06\pm 0.19$ & $44.89 \pm 0.42$ & $10.95\pm 0.02$ & $10.97 \pm 0.05$ & $3.72\pm 0.01$ & $3.72\pm 0.02$ \\
     \textsc{HGN (Leapfrog)} ours \textit{5-frame inference}& $12.15 \pm 0.05$ & $12.21 \pm 0.11$ & $44.29\pm 0.19$ & $44.12\pm 0.42$ & $6.28 \pm 0.03$ & $6.33 \pm 0.06$ & $3.35\pm 0.01$ & $3.35\pm 0.02$ \\
     \Xhline{3\arrayrulewidth}
    
    \end{tabular}}
    \vspace{0.25cm}
    \caption{Average pixel MSE of the reconstruction of a 30-frame rollout sequence on the test and train datasets of the four physical systems presented by \cite{hgn}. All the values are multiplied by $10^4$. We show our results (second and third group) along with the ones reported by the original authors (first group). In the second group, we train to reconstruct the whole inputted sequence (as an autoencoder) and in the third group, we train by inputting only the first 5 frames.}
    \label{tab:reproduction}
\end{table}

Table \ref{tab:reproduction} shows the results of the experiments described previously along with the results of the original authors. As it can be seen, we achieve average pixel reconstruction errors that are similar (30\% avg absolute error w.r.t. the reported values on the test set using Leapfrog integrator) to the ones reported in the original paper when reconstructing the same sequence that is inputted (we call this version \textit{autoencode}).
However, when attempting to train to reconstruct a rollout given only the first 5 frames our model performs poorly, with 107\% average absolute error on the test set, using Leapfrog integrator.
% This might be for several reasons that will be discussed in Sec. \ref{sec:disc}.
%In general, we can observe that our HGN and the proposed modifications learned well on the four physical systems.
% As it can be observed, the results reported by both versions of the HNN are one order of magnitude higher than the four versions of the HGN. Visual inspections of the results provided by HNN show that the network simply learned to output a static image.

\begin{figure}
    \centering
    \begin{subfigure}{.48\textwidth}
        \centering
        \includegraphics[width=.9\linewidth]{pictures/rollout_samples/new_forward_unroll_2_body.png}
        \label{fig:rollout-3-body}
        \caption{}
    \end{subfigure}
    \begin{subfigure}{.48\textwidth}
        \centering
        \includegraphics[width=.9\linewidth]{pictures/rollout_samples/new_forward_backward_unroll_pendulum.png}
        \caption{}
        \label{fig:rollout-pendulum}
    \end{subfigure}
    \caption{(a) Reconstruction of a sequence of the 2-body system along with a backward unroll of the data from the final state, and a forward rollout of the HGN trained using state inference from the first 5 frames. (b) Reconstruction of a sequence of the pendulum system along with a sped up and a slowed down forward rollout.}
    \label{fig:rollout_back_forth}
\end{figure}

In Figure \ref{fig:rollout_back_forth}, we show some qualitative examples of the reconstructions obtained by the full version of HGN. The model can reconstruct the samples and its rollouts can be reversed in time, sped up, or slowed down by changing the value of the time step used in the integrator. Since the HGN is designed as a generative model, we can sample from the latent space to produce initial conditions and perform their time evolution. We show some rollouts obtained this way in figure \ref{fig:samples}. We observe that our model is only able to generate plausible and diverse samples in the mass-spring dataset.
This behavior is different than the one shown by \cite{hgn} and might be caused by different hyper-parameter configurations in the training procedure or some implementation mistake.
% Later discussion with the authors revealed that they did not input the whole rollout to the encoder but just the first five frames, which was not explained in the paper. Since we received the information too late, we could not run all the experiments again.
% However, we tried this modification in the \textit{mass-spring} dataset, achieving a reconstruction loss of $1.4\times 10^{-3}$, which is very similar to authors results. 


% - The way we assess error: larger bodies with more movement gets more penalized
% - 2-3 bodies move slower and seems better for our model
% - Seems that our hyper-parameter choice is better at physics and theirs better at reconstruction
% - Identify good/bad physiscs and good/bad reconstructions


% Below we comment on the results reported for each environment.

% We takl about mass spring. then talk about pendulum. 
We achieve slightly larger MSE in the autoencode version and significantly larger in the 5-frame inference problem on both the mass-spring and pendulum.
The latter presents roughly double MSE probably because of a wider span of movement.
In general, these two environments show worse results in comparison to two/three-bodies.
For these last cases, our implementation using the \textit{autoencode} setting outperforms the original HGN \cite{hgn}, and when using the \textit{5-frame inference} the results are similar.
As we can see, these two environments show much less average pixel MSE compared to the first ones (almost one order of magnitude).
We believe this may be due to the differences when rendering the instances of each dataset.
The elements appearing in mass-spring and pendulum (represented by a large yellow ball) are larger than the ones present in the two/three bodies (two/three small coloured balls).
Because of this, it would be reasonable to assume that localization errors are more penalized in the first two environments, since the total difference in areas is larger.
Furthermore, the dynamics representing mass-spring and pendulum show faster movements in comparison to two/three-bodies, resulting in being harder to represent with our HGN.
Consequently, we hypothesise the following: larger elements and faster dynamics, produces higher average MSE on our model regardless of the difficulty of the environment physics.
However, this is not the case for the original author's results, who seem to struggle more on the two/three-bodies.
Surprisingly, it seems that our hyperparameter and architecture choices led to poorer reconstruction capabilities (higher MSE) but learning better physics (qualitatively more realistic movements).
% The error gets severely exacerbated when only inputting the first 5 frames of the sequence.

% In fact, 2 bodies are tal tal ,3 bodies are tal tal, and it is because of the previous assumption.






% \begin{itemize}
%     \item \textbf{Mass-spring and Pendulum:}

    
%     In comparison, both environments seem to be harder for us to learn than the multiple bodies ones.
     
%     % Surprisingly, it seems that our hyperparameter and architecture choices let to learning better physics but poorer reconstruction capabilities.

%     \item \textbf{Two-body and Three-body:} As we observe, our implementation using the \textit{autoencode} setting is able to outperform the original HGN \cite{hgn}. On the other hand, when using \textit{5-frame inference}, the results are still comparable. 
    
%     %Although it is hard to draw an exact conclusion on the factors causing these results, we can see the instances represented in the \textit{two/three-body} dataset are smaller compared to the ones in the \textit{mass-spring} or \textit{pendulum} (see Figure \ref{fig:datasets} for reference). Moreover, the dynamics of the \textit{two/three-body} systems are steadier. These two factors
    
%     Autoencode is better than them, 5-frames is similar
%     \item \textbf{Three-body:} Both autoencode and 5-frame better than them. Why?
% \end{itemize}


\begin{figure}
    \centering
    \includegraphics[width=\textwidth]{pictures/rollout_samples/new_sampling.png}
    \caption{Examples of sample rollouts from the latent space for different physical systems.}
    \label{fig:samples}
\end{figure}

\subsection{Additional experiments} \label{sec:additional_experiments}

\paragraph{GECO parameter search} \label{sec:hyperparam_search} The paper does not provide the values of GECO \cite{geco} used. In GECO, the Lagrangian multiplier is optimized at each step with a rate $\gamma$.
Figure \ref{fig:geco_search} shows the behavior of GECO for $\gamma \in \{0.1, 0.05, 0.01\}$ in terms of reconstruction loss and KL divergence. Higher values of $\beta$ give a better reconstruction loss but greatly increase the KL divergence. %In our experiments we generally use $\beta = 0.05$, which provides a good trade-off, or $\beta = 0.1$ for environments in which reconstruction is harder (e.g. pendulum).
However, we found that hyperparameters were not consistent among different environments and integrators. For this reason, we do not use GECO in our experiments.
\begin{figure}[]
\minipage{0.5\textwidth}
\centering
  \includegraphics[width=0.8\linewidth]{pictures/parameter_comparisons/lagrange_multiplier_comparison_rec_loss.png}
\endminipage\hfill
\minipage{0.5\textwidth}
\centering
  \includegraphics[width=0.8\linewidth]{pictures/parameter_comparisons/lagrange_multiplier_comparison_kld.png}
\endminipage
\caption{Reconstruction loss and KL divergence for different GECO parameters in the Pendulum environment.}
\label{fig:geco_search}
\end{figure}


\paragraph{Integrators} \label{sec:integrators}
Performing the integration step is key to generate the time evolution of a rollout given the initial state. 
In the HGN paper \cite{hgn} the system is tested using Euler and Leapfrog integration. We wonder if using higher order integration methods might boost the performance of the rollout generation process.
% The Runge-Kutta integration is only used when training the HNN architecture \cite{hnn}. 
%We observed that one main disadvantage of Euler integration is that its errors accumulate rapidly over longer time periods.
Therefore, we implement and test the HGN architecture with two additional numerical integration methods: the Runge-Kutta's 4th-order integrator \cite{rk4} and the 4th-order Leapfrog integrator (Yoshida's algorithm \cite{yoshida1992symplectic}). Table \ref{tab:integrators} shows a comparison of all four integrators on the Pendulum dataset.
%For this reason other methods that involve more integration steps, such as the fourth-order Runge-Kutta integration (RK4) or the fourth-order Leapfrog (Yoshida's algorithm) \cite{yoshida1992symplectic} have been implemented.
Both Leapfrog and Yoshida are \textit{symplectic} integrators: they guarantee to preserve the special form of the Hamiltonian over time \cite{neal2011mcmc}.
% , at the cost of slower execution.

\begin{table}[]
    \centering

    \begin{tabular}{c c c c c}
     \Xhline{3\arrayrulewidth}
      & \textsc{Euler} & \textsc{Runge-kuta 4} & \textsc{Leapfrog} & \textsc{Yoshida}\\

     \Xhline{3\arrayrulewidth}
     pixel MSE & $17.86 \pm 0.13$ & $76.88 \pm 0.08$ & $14.97\pm 0.10$ & $14.70\pm 0.10$ \\
     $\mathcal{H}$ std & $3.81$ & $0$ & $1961.93$ & $1893.05$\\
     reconstr. time (s) & $0.32$ & $1.89$ & $0.96$ & $1.61$ \\

     \Xhline{3\arrayrulewidth}
    
    \end{tabular}
    \vspace{0.25cm}
    \caption{Comparison between four different integrators used to perform the time evolution in the HGN. The results are measured on the simple pendulum test set. The pixel MSE values have been multiplied by $10^4$.}
    \label{tab:integrators}
\end{table}

Table \ref{tab:integrators} shows the average pixel MSE, the averaged standard deviation of the output of the Hamiltonian network during testing, and the reconstruction time of a single batch ($\texttt{batch}=16$) using the different integration methods that we have described previously. The model has been trained on the simple pendulum dataset. As we can see, the reconstruction time increases when using higher-order integration methods, since they require more integration steps. In general, we see that Euler integration offers a fast and sufficiently reliable reconstruction of the rollouts. Moreover, we observe that the fourth-order symplectic integrator (Yoshida) achieves the best performance. Surprisingly, the symplectic integration methods show more variance in the output of the Hamiltonian networks throughout a single rollout. This behavior is unexpected since using a symplectic integration method should ideally keep the value of the Hamiltonian invariant. We conclude that more experiments need to be performed to guarantee that the implementation of both Leapfrog and Yoshida integration methods are faithful to their formulation.


\paragraph{Integrator modelling} We train the modified architecture of Section \ref{sec:integrator_modelling} on the Pendulum dataset for 5 epochs. The architecture is the same as HGN, but the Hamiltonian Network now outputs $\Delta q$ and $\Delta p$. The average MSE error over the whole Pendulum dataset is $1.485\times 10^{-3}$, while in the test set it is $1.493 \times 10^{-3}$, which are both very close ($\sim \pm 2\%$) to those of autoencoding HGN (see Table \ref{tab:reproduction}). The modified architecture is still capable of performing forward slow-motion rollouts by modifying $\Delta t$. We set $\Delta t' = \frac{\Delta t}{2}$ and we compute the average MSE of the slow-motion reconstruction over 100 rollouts. The modified architecture achieved an error of $8$x$10^{-4}$, while the standard HGN achieved $9$x$10^{-4}$. Note that reconstruction losses are smaller for slow-motion as the images change less between timesteps.



\paragraph{Extra environments}
Apart from the four physical systems presented by \cite{hgn} we test our re-implementation of the HGN with physical systems that do not have a simple Hamiltonian expression. As described previously, these are the damped harmonic oscillator and the double pendulum. On one hand, we are interested in a damped system since it introduces a dissipative term to the equations of motion; a feature that differs from the previous systems. On the other hand, the double pendulum is modelled by a non separable Hamiltonian: $\mathcal{H}(\textbf{q},\textbf{p}) \neq K(\textbf{p}) + V(\textbf{q})$ as described previously. In figure \ref{fig:extra} we show some visual examples of the reconstructions provided by the HGN trained on the two systems. As we can see, HGN is able to reconstruct the damped oscillator with high reliability. Regarding the double pendulum, we observe that the model reconstructs well small oscillations, but fails when the trajectory is too chaotic as expected. The average pixel MSE of the reconstructions of the damped oscillator and the double pendulum are $6.39\cdot 10^{-4}$ and $6.91\cdot 10^{-4}$ respectively. The HGN is able to provide better reconstructions for these systems in comparison to the mass-spring and pendulum systems.

\begin{figure}
    \centering
        \begin{subfigure}{.48\textwidth}
        \centering
        \includegraphics[width=\linewidth]{pictures/rollout_samples/new_chaotic_pendulum_rollouts.png}
        \label{fig:a}
    \end{subfigure}
    \begin{subfigure}{.48\textwidth}
        \centering
        \includegraphics[width=\linewidth]{pictures/rollout_samples/new_damped_spring_sample_rollout.png}
        \label{fig:b}
    \end{subfigure}
    \caption{Examples of reconstructions of the double pendulum (left) and the damped harmonic oscillator (right).}
    \label{fig:extra}
\end{figure}

\section{Discussion}
\label{sec:disc}
% Give your judgement on if you feel the evidence you got from running the code supports the claims of the paper. Discuss the strengths and weaknesses of your approach - perhaps you didn't have time to run all the experiments, or perhaps you did additional experiments that further strengthened the claims in the paper.

% Summary
We were able to implement and train an Hamiltonian Generative Network with similar reconstruction performance of the ones of the original paper  ($30\%$ average absolute relative error wrt to their reported values when treating it as an autoencoder). These results show that the network is capable of exploiting the Hamiltonian equations to learn dynamics of a physical system from RGB images. However, the value of the resulting Hamiltonian does not remain constant throughout the system evolution. 
% \todo{Carles, Oleguer, check that what I write here makes sense}
This means that the network is learning something that is different from the Hamiltonian equations described in Section \ref{sec:data}.
%It is worth to note that there is an infinite number of equations whose partial derivatives in $q$ and $p$ match those of Eq. \ref{eq:hamilton}. In particular, a constant could be added to all Hamiltonian equations of Section \ref{sec:data} without changing the derivatives used by the integrators. Therefore, at each time-step the Hamiltonian network could predict any of these equations, explaining the high variance in the Hamiltonian values.   

To make the variational sampling work, we tried performing a grid search on the Geco\cite{geco} hyperparameters and using a fixed Lagrange multiplier as in \cite{beta-vae}. However, despite our best efforts, the samples produced by the variational model have very poor quality. This is generally due to the difficulty in minimizing both KL divergence and reconstruction loss. 

%To better evaluate the system, we provided further experiments giving a better grasp of the possibilities the Hamiltonian paradigm opens when using ANNs to learn about dynamical systems.
% Future work
We believe that further experiments are needed to understand better the behavior of the system and to improve it. Future work could include further testing on each network architecture, probably smaller networks would also be able to encode the needed information. Another next step is to try the approach on more challenging (and realistic-looking) environments. In addition, it would be interesting to tackle the transfer learning capabilities of such architecture between different environments. How re-usable each network is? How much faster the system is able to learn the new dynamics? Finally, another field which could benefit from this research is model-based reinforcement learning.
A generative approach from which to sample example rollouts could be very useful for training agents without the need of directly interacting with the environment.

\subsection{What was easy}
% Give your judgement of what was easy to reproduce. Perhaps the author's code is clearly written and easy to run, so it was easy to verify the majority of original claims. Or, the explanation in the paper was really easy to follow and put into code. 

% Be careful not to give sweeping generalizations. Something that is easy for you might be difficult to others. Put what was easy in context and explain why it was easy (e.g. code had extensive API documentation and a lot of examples that matched experiments in papers). 
Once we implemented the code it resulted quite easy to perform multiple experiments on different environments, architectures and hyper-parameters due to the code's modularity and flexibility.
We can define the the previously mentioned experiments and most common testing behaviors from a set of yaml files which can then be modified from command-line arguments.
While this required extra planning and work at the beginning it really payed off when debugging and evaluating in later stages. 

\subsection{What was difficult} \label{sec:challenge}
% List part of the reproduction study that took more time than you anticipated or you felt were difficult. 

% Be careful to put your discussion in context. For example, don't say "the maths was difficult to follow", say "the math requires advanced knowledge of calculus to follow". 

The main challenge we encountered is finding the correct tools to debug a model composed of so many interconnected networks.
The fact that it has a variational component with a dynamic Lagrange multiplier term makes it especially tricky to train.
Furthermore, no public implementation existed and some details and parameters were missing in the original paper leading to some necessary assumptions or parameter searches.


\subsection{Communication with original authors}
% Document the extent of (or lack of) communication with the original authors. To make sure the reproducibility report is a fair assessment of the original research we recommend getting in touch with the original authors. You can ask authors specific questions, or if you don't have any questions you can send them the full report to get their feedback before it gets published. 
We first tried to understand and re-implement the code by ourselves.
Nevertheless, at some point we had gathered a significant set of doubts and we decided to email them to the original authors, which they answered with great detail.
From that point onwards, we sent a couple more set of doubts, also receiving answers.\\

Most of our doubts were about network architecture clarifications (either of unclear or missing descriptions from the original paper), and loss function evaluation.
Furthermore, they provided us with some of their environment images so we could more easily make our environments as similar as possible.

\subsection{Improving reproducibility}
Having worked in re-implementing the whole original work, we feel it is important to share our experience as well as providing a recommendation on how it could be made more easily reproducible.
First, having the environments data or code to generate it available online would save the effort and, most importantly, it would constitute a baseline against which to compare future work.
Secondly, publishing all the hyperparameters and more details of the networks architecture would make the whole work much easier to reproduce and require less training attempts, especially for what concerns GECO.


\section*{Acknowledgements}
We thank Stathi Fotiadis for voluntarily contributing with a GECO \cite{geco} implementation draft to the public \href{https://github.com/CampusAI/Hamiltonian-Generative-Networks}{repo} and his useful feedback on code structuring. We thank the KTH Robotics, Perception, and Learning (RPL) Lab for the computational resources provided to us.
In addition, we would like to thank the original authors for providing further details on the implementation.
