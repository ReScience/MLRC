% DO NOT EDIT - automatically generated from metadata.yaml

\def \codeURL{https://github.com/elfrink1/FACT}
\def \codeDOI{}
\def \codeSWH{swh:1:dir:445130f59283e6dce7df5eb72dd346a5c57c9230}
\def \dataURL{}
\def \dataDOI{}
\def \editorNAME{Koustuv Sinha}
\def \editorORCID{}
\def \reviewerINAME{Anonymous Reviewers}
\def \reviewerIORCID{}
\def \reviewerIINAME{}
\def \reviewerIIORCID{}
\def \dateRECEIVED{29 January 2021}
\def \dateACCEPTED{01 April 2021}
\def \datePUBLISHED{26 May 2021}
\def \articleTITLE{[Re] Explaining Groups of Points in Low-Dimensional Representations}
\def \articleTYPE{Replication}
\def \articleDOMAIN{ML Reproducibility Challenge 2020}
\def \articleBIBLIOGRAPHY{bibliography.bib}
\def \articleYEAR{2021}
\def \reviewURL{https://openreview.net/forum?id=cqAHExg2f}
\def \articleABSTRACT{This report covers our reproduction of the paper Explaining Low dimensional Representation by Plumb et al. In this paper, a method (Transitive Global Translations, TGT) is proposed for explaining different clusters in low dimensional representations of high dimensional data. They show their method outperforms the Difference Between the Means (DBM) method, is consistent in explaining differences with few features and matches real patterns in data. We verify these claims by reproducing their experiments and testing their method on new data. We also investigate the use of more complex transformations to explain differences between clusters.}
\def \replicationCITE{Gregory Plumb, Jonathan Terhorst, Sriram Sankararaman, Ameet Talwalkar; Proceedings of the 37th International Conference on Machine Learning, PMLR 119:7762-7771}
\def \replicationBIB{plumb2020explaining}
\def \replicationURL{http://proceedings.mlr.press/v119/plumb20a.html}
\def \replicationDOI{}
\def \contactNAME{Rajeev Verma}
\def \contactEMAIL{rajeev.verma@student.uva.nl}
\def \articleKEYWORDS{rescience c, rescience x}
\def \journalNAME{ReScience C}
\def \journalVOLUME{7}
\def \journalISSUE{2}
\def \articleNUMBER{}
\def \articleDOI{}
\def \authorsFULL{Rajeev Verma et al.}
\def \authorsABBRV{R. Verma et al.}
\def \authorsSHORT{Verma et al.}
\title{\articleTITLE}
\date{}
\author[1,\orcid{0000-0002-2340-0942}]{Rajeev Verma}
\author[1]{Jim J. O. Wagemans}
\author[1,\orcid{0000-0003-2344-2787}]{Paras Dahal}
\author[1]{Auke Elfrink}
\affil[1]{University of Amsterdam, The Netherlands}
