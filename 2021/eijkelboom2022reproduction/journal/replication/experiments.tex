\documentclass{article}
\usepackage[utf8]{inputenc}
\usepackage{amsmath}
\usepackage{amsfonts}
\usepackage{url}
\usepackage{todonotes}
\DeclareMathOperator*{\argmin}{arg\,min}

\usepackage{geometry}
 \geometry{
 a4paper,
 total={170mm,257mm},
 left=20mm,
 top=20mm,
 }
\begin{document}

{\Large \textbf{Ideas for replication}}

\bigskip

\textbf{Experiments}

\begin{enumerate}
    \item Reproduction, do we get the same results as the authors? Moreover, provide confidence intervals to interpret the results better.
    \item Perform similar experiments on different datasets. Compare if they work, and if they are significantly better than previous SotA. Datasets we will consider:
    \begin{itemize}
        \item Dutch census dataset \url{https://www.cbs.nl/nl-nl/achtergrond/2005/45/the-dutch-national-census-2001--40-excel-tables--} Checken!!!
        \item Student-Mathematics dataset: \url{https://www.kaggle.com/janiobachmann/math-students}
    \end{itemize}
    \item Consider different cluster algorithms. Algorithms we will consider:
    \begin{itemize}
        \item K-center 
        \item PAM (partition clustering)
        \item Clara (partition clustering)
        \item DBSCAN (density based)
    \end{itemize}
    \item Alter $J$ values, compare Bank dataset but with $J = 2$ (instead of $J=3$ martial status) as used in Bera et al. 2019 and Backurs et al. 2019 and see if Variational Fair clustering still performs better. 
    \item Extra: Experiment with different $K$/$U$. What if $U$ is not accurate/biased itself. 
    \item Andere divergence.
\end{enumerate}



\end{document}
