% DO NOT EDIT - automatically generated from metadata.yaml

\def \codeURL{https://github.com/Veranep/rideshare-replication}
\def \codeDOI{10.5281/zenodo.6501799}
\def \codeSWH{swh:1:dir:f5439c1a7a15c4eb709da6f32eb252679a1d44bd}
\def \dataURL{https://www1.nyc.gov/site/tlc/about/tlc-trip-record-data.page}
\def \dataDOI{}
\def \editorNAME{Koustuv Sinha,\\ Sharath Chandra Raparthy}
\def \editorORCID{}
\def \reviewerINAME{Anonymous Reviewers}
\def \reviewerIORCID{}
\def \reviewerIINAME{}
\def \reviewerIIORCID{}
\def \dateRECEIVED{04 February 2022}
\def \dateACCEPTED{11 April 2022}
\def \datePUBLISHED{19 May 2022}
\def \articleTITLE{[Re] Replication study of 'Data-Driven Methods for Balancing Fairness and Efficiency in Ride-Pooling'}
\def \articleTYPE{Replication}
\def \articleDOMAIN{ML Reproducibility Challenge 2021}
\def \articleBIBLIOGRAPHY{bibliography.bib}
\def \articleYEAR{2022}
\def \reviewURL{https://openreview.net/forum?id=BEhgn2zm3CK}
\def \articleABSTRACT{We evaluate the following claims related to fairness-based objective functions presented in the original work: (1) For the four objective functions, the success rate in the worst-off neighborhood increases monotonically with respect to the overall success rate. (2) The proposed objective functions do not lead to a higher income for the lowest-earning drivers, nor a higher total income, compared to a request-maximizing objective function. (3) The driver-side fairness objective can outperform a request-maximizing objective in terms of overall success rate and success rate in the worst-off neighborhood. We evaluate the claims by the original authors by (a) replicating their experiments, (b) testing for sensitivity to a different value estimator, (c) examining sensitivity to changes in the preprocessing method, and (d) testing for generalizability by applying their method to a different dataset. We reproduced the first claim since we observed the same monotonic increase of the success rate in the worst-off neighborhood with respect to the overall success rate. The second claim we did not reproduce, since we found that the driver-side fairness objective function obtains a higher income for the lowest-earning drivers than the request-maximizing objective function. We reproduced the third claim, since the driver-side objective function performs best in terms of overall success rate and success rate in the worst-off neighborhood, and also reduces the spread of income. Changes of the value estimator, preprocessing method and even dataset all led to consistent results regarding these claims.}
\def \replicationCITE{Naveen Raman, Sanket Shah, John P. Dickerson. Data-Driven Methods for Balancing Fairness and Efficiency in Ride-Pooling (IJCAI 2021).}
\def \replicationBIB{raman2021data}
\def \replicationURL{https://arxiv.org/pdf/2110.03524.pdf}
\def \replicationDOI{10.48550/arXiv.2110.03524}
\def \contactNAME{Vera Neplenbroek}
\def \contactEMAIL{vera.neplenbroek@student.auc.nl}
\def \articleKEYWORDS{rescience c, machine learning, deep learning, python, pytorch, ridesharing, fairness}
\def \journalNAME{ReScience C}
\def \journalVOLUME{8}
\def \journalISSUE{2}
\def \articleNUMBER{29}
\def \articleDOI{10.0000/zenodo.0000000}
\def \authorsFULL{Vera Neplenbroek, Sabijn Perdijk and Victor Prins}
\def \authorsABBRV{V. Neplenbroek, S. Perdijk and V. Prins}
\def \authorsSHORT{Neplenbroek, Perdijk and Prins}
\title{\articleTITLE}
\date{}
\author[1,2,\orcid{0000-0003-0158-6608}]{Vera Neplenbroek}
\author[1,2,\orcid{0000-0002-6771-0985}]{Sabijn Perdijk}
\author[1,2,\orcid{0000-0001-7645-3119}]{Victor Prins}
\affil[1]{University of Amsterdam, Amsterdam, the Netherlands}
\affil[2]{Equal contributions}
