% DO NOT EDIT - automatically generated from metadata.yaml

\def \codeURL{https://github.com/MarusaOrazem/reproducibility_challenge}
\def \codeDOI{10.5281/zenodo.6505384}
\def \dataURL{}
\def \dataDOI{}
\def \editorNAME{}
\def \editorORCID{}
\def \reviewerINAME{}
\def \reviewerIORCID{}
\def \reviewerIINAME{}
\def \reviewerIIORCID{}
\def \dateRECEIVED{01 November 2018}
\def \dateACCEPTED{}
\def \datePUBLISHED{}
\def \articleTITLE{[Re] Graph Edit Networks}
\def \articleTYPE{Editorial}
\def \articleDOMAIN{Graph Machine Learning}
\def \articleBIBLIOGRAPHY{bibliography.bib}
\def \articleYEAR{2022}
\def \reviewURL{}
\def \articleABSTRACT{he studied paper proposes a novel output layer for graph neural networks (the graph
edit network ‐ GEN). The objective of this reproduction is to assess the possibility of
its re‐implementation in the Python programming language and the adherence of the
provided code to the methodology, described in the source material. Additionally, we
rigorously evaluate the functions used to create the synthetic data sets, on which the
models are evaluated. Finally, we also pay attention to the claim that the proposed ar‐
chitecture scales well to larger graphs.}
\def \replicationCITE{Paassen, Benjamin, et al. "Graph edit networks." International Conference on Learning Representations. 2020.}
\def \replicationBIB{paassen2020graph}
\def \replicationURL{https://openreview.net/pdf?id=dlEJsyHGeaL}
\def \replicationDOI{}
\def \contactNAME{Vid Stropnik}
\def \contactEMAIL{vs2658@student.uni-lj.si}
\def \articleKEYWORDS{machine learning, network analysis, graph machine learning, pytorch, python}
\def \journalNAME{ReScience C}
\def \journalVOLUME{4}
\def \journalISSUE{1}
\def \articleNUMBER{}
\def \articleDOI{}
\def \authorsFULL{Vid Stropnik and Maruša Oražem}
\def \authorsABBRV{V. Stropnik and M. Oražem}
\def \authorsSHORT{Stropnik and Oražem}
\title{\articleTITLE}
\date{}
\author[1,*,\orcid{0000-0002-5848-9596
}]{Vid Stropnik}
\author[1,\orcid{0000-0001-8760-4234}]{Maruša Oražem}
\affil[1]{Faculty of Computer and Information Science, University of Ljubljana, Ljubljana Slovenia}

