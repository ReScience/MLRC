% DO NOT EDIT - automatically generated from metadata.yaml

\def \codeURL{https://github.com/Di-ayy-go/fact-ai}
\def \codeDOI{10.5281/zenodo.6518051}
\def \codeSWH{swh:1:dir:45176f5005ed390a349cd01e61ed37711095879e}
\def \dataURL{}
\def \dataDOI{}
\def \editorNAME{}
\def \editorORCID{}
\def \reviewerINAME{Anonymous Reviewers}
\def \reviewerIORCID{}
\def \reviewerIINAME{}
\def \reviewerIIORCID{}
\def \dateRECEIVED{15 February 2020}
\def \dateACCEPTED{15 February 2020}
\def \datePUBLISHED{21 May 2020}
\def \articleTITLE{[Re] Replication Study of "Fairness and Bias in Online Selection"}
\def \articleTYPE{Replication}
\def \articleDOMAIN{ML Reproducibility Challenge 2021}
\def \articleBIBLIOGRAPHY{bibliography.bib}
\def \articleYEAR{2020}
\def \reviewURL{}
\def \articleABSTRACT{ In this paper, we work on reproducing the results obtained in the 'Fairness and Bias in Online Selection' paper \citep{correa2021fairness}. The goal of the reproduction study is to validate the 4 main claims made in \cite{correa2021fairness}. The claims made are: (1) for the multi-color secretary problem, an optimal online algorithm is fair, (2) for the multi-color secretary problem, an optimal offline algorithm is unfair, (3) for the multi-color prophet problem, an optimal online algorithm is fair (4) for the multi-color prophet problem, an optimal online algorithm is less efficient relative to the offline algorithm.
To test if the results of the secretary algorithm generalize to other data sets, the proposed algorithms and baselines are applied to the UFRGS Entrance Exam and GPA data set \citep{DVN/O35FW8_2019}.

The paper that has been reproduced includes a link to a repository containing 	extit{C++} files for the algorithms that were implemented. For our experiments, we reimplemented the code in 	extit{Python}. Our goal was to reproduce the code in an efficient manner without altering the core logic. Using the Python code all the experiments in the paper have been replicated including some additional experiments to verify the claims made in \cite{correa2021fairness}.
The reproduced results support all claims made in \cite{correa2021fairness}. However, in the case of the unfair secretary algorithm (SA), some irregular results arise in the experiments due to randomness. This irregularity is also existent in the original code.
The concepts behind the algorithms were straightforward. The existing code base provided a solid reference point to verify the results of the original paper by compiling and running the provided code.
Implementing the prophet algorithm, in comparison to the secretary algorithm, was complex. 	extit{C++} is a more efficient compiler (time complexity, etc.) compared to Python. For the reproduction of the algorithms, this needed to be taken into account. While it might be possible to execute transliterated code on a powerful machine, with the available resources the code would have taken over 96 hours to run. In order to tackle this problem, some of the data structures needed to be converted to 	extit{NumPy} arrays to decrease computation time.}
\def \replicationCITE{J. Correa, A. Cristi, P. Duetting, and A. Norouzi-Fard. “Fairness and Bias in Online Selection.”}
\def \replicationBIB{correa2021fairness}
\def \replicationURL{https://www.dii.uchile.cl/~jcorrea/papers/Conferences/CCDF2021.pdf}
\def \replicationDOI{}
\def \contactNAME{Diego van der Mast}
\def \contactEMAIL{diego.vandermast@student.uva.nl}
\def \articleKEYWORDS{rescience c, rescience x, Python}
\def \journalNAME{None}
\def \journalVOLUME{}
\def \journalISSUE{}
\def \articleNUMBER{}
\def \articleDOI{}
\def \authorsFULL{Diego van der Mast et al.}
\def \authorsABBRV{D.V.D. Mast et al.}
\def \authorsSHORT{Mast et al.}
\title{\articleTITLE}
\date{}
\author[1,2,\orcid{0000-0002-0001-3069}]{Diego van der Mast}
\author[1,2,\orcid{0000-0003-4610-3542}]{Soufiane Ben Haddou}
\author[1,2,\orcid{0000-0003-1973-2749}]{Jacky Chu}
\author[1,2,\orcid{0000-0002-9430-2063}]{Jaap Stefels}
\affil[1]{Informatics Institute, University of Amsterdam, Amsterdam, Netherlands}
\affil[2]{Equal contributions}
